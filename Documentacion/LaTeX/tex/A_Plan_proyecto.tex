\apendice{Plan de Proyecto Software}

\section{Introducción}
En esta sección se detallará la planificación que se ha realizado, el estudio de viabilidad tanto de la parte económica como de la legal.

\section{Planificación temporal}
\subsection{Sprint 0 (26/10/2020 - 25/11/2020)}
Puesta a punto del proyecto, planteamiento de las herramientas con las que trabajar, búsqueda de alternativas y toma de contacto con las herramientas nuevas que se van a emplear.
Las tareas que se realizaron fueron:
\begin{itemize}
	\tightlist
	\item Añadir la extensión ZenHub al navegador
		Desde el \textbf{Chrome Web Store} de Google Chrome se añadió la extensión \textbf{ZenHub for GitHub}.
	\item Clonar el repositorio en local. 
		Mediante la aplicación \textbf{GitHub Desktop}, se clonó el repositorio del Gestor de TFG mediante el enlace HTTP que proporciona GitHub.
	\item Investigar sobre Vaadin.
		A través de la página oficial de \href{https://vaadin.com/}{Vaadin} se realizó la instalación y el aprendizaje acerca de Vaadin.
	\item Actualización del README.md 
		Se modificó el README.md del proyecto para que refleje los cambios respecto a la versión anterior. 
	\item Investigar LaTeX
		Se procedió a buscar información sobre cómo realizar el proceso de instalación y manejar \textbf{Latex} para realizar la documentación del proyecto posteriormente.
\end{itemize}

Se puede ver el trascurso de estas tareas gráficamente en la siguiente ilustración \ref{fig:Sprint0_Graficos}.

\imagenflotante{Sprint0_Graficos}{Gráfica Control chart- Sprint 0}{0.9}


\subsection{Sprint 1 (25/11/2020 - 12/01/2021)}
Generación de test unitarios, búsqueda de trabajos similares, cambio del driver para conectarse con el excel, información para obtener ideas de como realizar ciertas mejoras y comienzo de la documentación del proyecto. Mejora de la cobertura de la aplicación web.

Las tareas planteadas fueron:
\begin{itemize}
	\tightlist
	\item Instalación Miktex + TexStudio
		Tras la reunión del 25/11/20, se decidió que la mejor opción sería emplear un editor local, en lugar de usar \href{https://es.overleaf.com/}{Overleaf}, un editor online. Para ello, se instaló la distribución de TeX/LaTeX, \textbf{Miktex}, junto al editor de texto llamado \textbf{TexStudio}.
	\item Se comienza la documentación en LaTeX  - Sprint 0
		Creación de la memoria y los anexos en LaTeX a partir de las plantillas.
	\item Generar nuevos test
		Para aumentar la cobertura de la aplicación se definieron nuevos test.
	\item Creación carpeta pruebas
		Se creó la carpeta pruebas, dentro del apartado de documentación, donde se irán realizando pruebas antes de introducirlas en la aplicación. Dentro de esta carpeta se encontrarán aplicaciones prototipo, gracias a las cuales se logro una mejor comprensión del funcionamiento de Vaadin.
	\item Búsqueda de trabajos relacionados con la gestión de TFG/TFM
		Se realizó una investigación con el fin de encontrar proyectos similares a la aplicación web, es decir, que consistan en la gestión de trabajos de fin de grado o similares. Los proyectos encontrados serán explicados en el apartado \textbf{Trabajos relacionados} de la memoria.
	\item Realización de cambios sugeridos en la memoria y los anexos
		Se cambiaron algunos aspectos de la memoria y los anexos comentados en la reunión realizada el 15/12/20. Como la incorporación de diagramas de Zenhub para visualizar la evolución de las issues.
	\item Cambiar driver JDBC 
		Para poder usar ficheros en formato .ods se buscó una alternativa similar al driver para .csv que se emplea. Primero se opto por la opción de \textbf{\href{https://www.cdata.com/drivers/excel/jdbc/}{Microsoft Excel JDBC Driver}} con el cual se puede leer, escribir y actualizar Excel mediante JDBC. Sin embargo, está opción es de pago, por lo que fue descartada. Los drivers gratuitos para .ods que se encontraron fueron \href{https://odftoolkit.org/}{ODFDOM} y \href{http://www.jopendocument.org/}{JopenDocument} los cuales eran bastante viejos y en desuso. 
\end{itemize}

Se puede ver el trascurso a través de la siguiente imagen \ref{fig:Sprint1_Graficos}.

\imagenflotante{Sprint1_Graficos}{Gráfica Control chart- Sprint 1}{0.9}

\subsection{Sprint 2 (12/01/2021 - 26/01/2021)}
Búsqueda de nuevos drivers que implementen JDBC para ficheros .xls. Verificar el funcionamiento de las opciones encontradas para leer los .xls como \href{https://poi.apache.org/}{Apache POI}, \href{https://code.google.com/archive/p/sqlsheet/}{SQLSheet}, \href{https://codoid.com/fillo/}{API Fillo}, \href{https://www.cdata.com/drivers/excel/jdbc/}{Cdata JDBC for Excel} y JdbcOdbcDriver. Integrar la API Fillo en el proyecto. 

Las tareas planteadas fueron:
\begin{itemize}
	\tightlist
	\item Instalación del LibreOffice 
		Se instaló \href{https://es.libreoffice.org/}{Apache LibreOffice} para manipular los fichero .xls y .csv que se emplean para la parte de datos de la aplicación.
	\item Cambiar de formato al fichero ``\textbf{BaseDeDatosTFG.ods}''.
		Se modificó el formato del fichero ``\textbf{BaseDeDatosTFG.ods}'' a ``\textbf{BaseDeDatosTFG.xls}''.
	\item Anexos - Modificación para poder cambiar el tamaño de las imágenes.
		Cambios en la plantilla de \textbf{anexos.tex} para poder modificar el tamaño de las imágenes al deseado.
	\item Probar el Apache POI
		Para verificar el funcionamiento de Apache POI, se incluyo en el proyecto de prueba \textbf{HolaMundoVaadin} y se realizaron varios ejemplos de prueba tomando como referente el proyecto principal.
	\item Error al intentar ejecutar la imagen del proyecto gabrielstan/gestion-tfg
		Bug realizado al intentar desplegar el proyecto  gabrielstan/gestion-tfg.  
	\item Desplegar proyectos relacionados con la gestión de TFG
		Proceso por el cual se probó a desplegar los proyectos relacionados con \textbf{GII 20.09 Herramienta web repositorios de TFGII}. En el apartado de la memoria Trabajos relacionados se detallará sobre este tema. 
	\item Memoria - Mejoras
		Se realizaron varios cambios en la memoria.
	\item Incorporación del driver para fichero Excel
		Prueba de la \href{https://codoid.com/fillo/}{API Fillo} en el proyecto de prueba \textbf{HolaMundoVaadin}, y tras verificar su funcionamiento se procedió a incluirlo en el proyecto principal.
	\item Modificación de los nombres de los fichero csv 
		Se cambiaron los nombres de los ficheros .csv para que coincidieran con los nombres de las hojas del fichero .xls. 
	\item Memoria - Documentación Sprint 2
		Comienzo de la documentación de lo realizado en el Sprint 2. En el cual se detallan las tareas realizadas.
	\item Modificaciones en los test
		Para probar la incorporación de la \href{https://codoid.com/fillo/}{API Fillo} se modificaron el test  ``\textbf{\textbf{SintInfDataTest}}''.
	\item Realización de cambios para el funcionamiento de la nueva conexión para los fichero .xls 
		Para poder usar la \href{https://codoid.com/fillo/}{API Fillo} se han tenido que realizar multitud de cambios en el proyecto. Aunque tanto el ``\textbf{\textbf{CsvDriver}}'' y ``\textbf{\textbf{Fillo}}'' emplean lenguaje SQL, ``\textbf{CsvDriver}'' emplea JDBC puro mientras que ``\textbf{Fillo}'' usa funciones y clases propias.
\end{itemize}

La siguiente imagen \ref{fig:Sprint2_Graficos} muestra cómo se han desarrollado las tareas a lo largo del tiempo.

\imagenflotante{Sprint2_Graficos}{Gráfica Control chart- Sprint 2}{0.9}

\subsection{Sprint 3 (26/01/2021 - 11/02/2021)}

Las tareas planteadas fueron:
\begin{itemize}
	\tightlist
	\item Investigar opciones de hosting para el despliegue
		Se comparan varias opciones gratuitas para desplegar la aplicación, las cuales son: GitHub Pages, LucusHost, Awardspace, RunHosting, FreeHostia, X10Hosting y Heroku. Eligiendo la opción de \href{https://dashboard.heroku.com/}{Heroku}, entre otras razones, porque proporciona la opción de conectar el despliegue con GitHub. En la \href{https://github.com/dbo1001/Gestor-TFG-2021/issues/26}{issue de Github}, correspondiente a esta tarea, se encuentra detallado cada opción de despliegue.
	\item Rediseño fachada y vistas
		Se elimina la parte vinculada a los datos de las vistas incluyéndola en a las clases fachada. Consiguiendo que unicamente las clases fachadas comuniquen directamente con la capa de datos, y el resto de la aplicación emplee de intermediario a las clases fachada.
	\item Cambiar nombres a inglés.
		Se traducen los nombres de las variables, métodos y clases a inglés.
	\item Actualización de la Memoria y Anexos.
		Se realizan cambios y ampliaciones en el contenido de la documentación, en la Memoria y el Anexo.
	\item Instalación Heroku CLI
		Para realizar el despliegue del proyecto se instala el terminal de Heroku, denominado Heroku CLI. La instalación se llevo a cabo según el tutorial de instalación de la página oficial de \href{https://devcenter.heroku.com/articles/heroku-cli}{Heroku}.
	\item Incorporación del patrón Factory 
		Se crea una nueva clase con la función de seleccionar el tipo de acceso de datos, ya sea la clase fachada encargada de los ficheros .csv o, la que gestiona los ficheros .xls.
	\item Creación branch de prueba
		Rama del repositorio donde se almacena el proyecto de prueba formularioVaadin, el cual posteriormente será usado para realizar una prueba de despliegue en Heroku.
	\item Modificación de Test JUnit
		Se cambian los test para que sirvan para las funciones de la clase fachada para los ficheros .xls. 
	\item Probar el despliegue del proyecto de prueba
		Se emplea el proyecto de prueba, formularioVaddin, para realizar una prueba de despliegue en Heroku. Se intentará realizarlo tanto, a través de Heroku CLI como, mediante la interfaz la página de Heroku.
	
\end{itemize}

En la siguiente imagen se enseña gráficamente el desarrollo de las issues \ref{fig:Sprint3_Graficos}.

\imagenflotante{Sprint3_Graficos}{Gráfica Control chart- Sprint 3}{0.9}

\subsection{Sprint 4 (11/02/2021 - 23/02/2021)}
Se intentará incorporar la integración continua. Realizar el despliegue de la aplicación con Heroku. Añadir una columna de rankings en el Histórico. Cambiar las notas del histórico a privadas.

Las tareas planteadas fueron:
\begin{itemize}
	\tightlist
	\item Realizar la Integración Continua.
		A través de la opción \textbf{GitHub Actions} se incorporará la opción de compilar y ejecutar los test cuando se realiza un cambio (ya sea un push o un pull request) en el repositorio. Tiene como objetivo detectar fallos eficazmente mediante la integración automática frecuentemente de un proyecto. Se explica cómo se realizó en la \href{https://github.com/dbo1001/Gestor-TFG-2021/issues/37}{Github}.
	\item Actualización bibliografía
		Se incorporan los enlaces de las páginas o datos bibliográficos que se han empleado hasta ahora en la realización del proyecto.
	\item Cambiar las notas del Histórico a privadas
		Se cambiará la nota que figura en la tabla "Descripción proyectos" del Histórico para que no se muestre (como privada).
	\item Añadir dependencia Apache Commons Math
		Se empleará Apache Commons Math para calcular los percentiles de los ranking de las notas.
	\item Crear ranking de notas en Histórico
		Para sustituir la columna de notas se integrarán tres nuevas columnas con rankings.
	\item Actualización del despliegue del proyecto de prueba en Heroku 
		Se desplegó el proyecto principal, sistinf, en \textbf{Heroku}. Se podrá encontrar cómo realizar este proceso tanto en la \href{https://github.com/dbo1001/Gestor-TFG-2021/issues/36}{issue en Github} como en el apartado del \textbf{Manual del programador}.
	\item Mejora de test JUnit
		Se modificarán los tests existentes para aumentar la zona verificada por los tests.
	
\end{itemize}

Se puede apreciar gráficamente el desarrollo de las issues en la siguiente ilustración\ref{fig:Sprint4_Graficos}.

\imagenflotante{Sprint4_Graficos}{Gráfica Control chart- Sprint 4}{0.9}

\subsection{Sprint 5 (23/02/2021 - 09/03/2021)}
Añadir de nuevas columnas en la tabla de Históricos para reemplazar la columna Notas, la cual será eliminada. Crear test para comprobar los datos de los ficheros XLS. Investigación y prueba de análisis de la calidad del código con \href{https://sonarcloud.io/}{SonarCloud}.

Las tareas realizadas fueron:
\begin{itemize}
	\tightlist
	\item Eliminación de la columna Nota del Histórico
		Debido al carácter sensible de las notas, se eliminarán de la tabla de la vista del Histórico.
	\item Crear columna del ranking total en el Histórico. 
		Se creó una nueva columna en la tabla de la vista del Histórico para representar las notas en comparación al resto de ellas.
	\item Investigar sobre \textbf{SonarCloud}. 
		Para realizar las métricas que posteriormente irán en el apartado de Métricas, se analizará la calidad del código de algunos de los proyectos realizados anteriormente. Para ello, se investigará como usar el software \href{https://sonarcloud.io/}{SonarCloud} para realizar las mediciones de calidad. Consta de dos versiones, una de pago y otra gratuita, siendo está última la que se empleará. Al ser una versión gratuita los proyectos serán públicos. 
	\item Crear columna del ranking por cursos en el histórico. 
		Al igual que con el caso del ranking total, se añadirá una nueva columna en la tabla de Históricos que contenga el ranking de una nota con respecto al resto de notas de ese mismo curso escolar (1 de septiembre a 30 de junio). 
	\item Probar a desplegar el proyecto en Linux.
	 	Al estar el proyecto desplegado en \textbf{Heroku} se podrá acceder a él a través de la url, donde se encuentra el proyecto, desde cualquier SO.
	\item Parsear ficheros CSV. 
		Se añadirán nuevos test que verifiquen que no existan errores de formato, por ejemplo que el formato de las fechas sea el indicado
	\item Separación de la clase SintInfDataTest en \textbf{SintInfDataTestXLS} y \textbf{SintInfDataTestCSV}.
		 Al añadir una nueva clase fachada para el tipo de datos XLS se necesita otra clase para testarla, por lo que se dividirá la clase SintInfDataTest en dos clases, SintInfDataTestXLS, encargada de verificar los archivos y funciones correspondientes a los datos en XLS, y SintInfDataTestCSV, la cual testará los ficheros CSV.
	 \item Añadir botón quality gate SonarCloud. 
		 Se incorporará en el README.md del proyecto un acceso directo a la página donde figurarán los análisis de los proyectos en SonarCloud. 
	 \item Añadir botón despliegue Heroku.
	  	Al igual que con SonarCloud, se incluirá un acceso a la página donde se encuentra desplegado el proyecto en Heroku. Para realizarlo se siguió los pasos de la documentación de \href{https://devcenter.heroku.com/articles/heroku-button}{Heroku}.
	 \item Analizar el proyecto sistinf con SonarCloud. 
	 	Se realizará el análisis automático del proyecto principal, en el cual no se incluye el análisis del código en Java ya que esté lenguaje no está soportado en \href{https://sonarcloud.io/}{SonarCloud}.
	 \item Incluir el análisis automático del proyecto en Sonarcloud. 
	 	Se seguirá el siguiente tutorial de la página de \href{https://sonarcloud.io/}{SonarCloud} para que se realice el análisis cada vez que se realice un push.  
	 \item Anexos actualización – B-Requisitos. 
	 	Se añadirá el apartado de requisitos a la documentación de LaTex
	 \item Anexos actualización - A-Plan-Proyecto.
	  	Modificación de los Anexos con el apartado A-Plan-Proyecto.	
	
\end{itemize}

En la siguiente gráfica se puede ver el desarrollo de las tareas del Sprint\ref{fig:Sprint5_Graficos}.

\imagenflotante{Sprint5_Graficos}{Gráfica Control chart- Sprint 5}{0.9}

\subsection{Sprint 6 (09/03/2021 - 23/03/2021)}
Análisis del proyecto poolobject del código en Java en SonarCloud. Se investigó y probó a realizar el login a través de distintos métodos.

Las tareas realizadas fueron:
\begin{itemize}
	\tightlist
	\item Análisis del proyecto de prueba poolobject con SonarCloud.
		Se trata de un proyecto en Java al igual que el del Gestor-TFG-2021. Para ello, se realizará un fork del proyecto en Github y se procederá a realizar el análisis especificando la ubicación del código fuente en Java y los archivos binarios. Se debe realizar de esta forma debido a que en el análisis automático de \href{https://sonarcloud.io/}{SonarCloud} no se incluye Java.
	\item Pruebas de Login a través de Heroku. 
		Se probó a acceder a la parte de actualización de ficheros a través del login y subir un fichero, pero, no se modificaron los datos de las vistas ni la fecha de actualización.
	\item Prueba Login con la app desplegada con Tomcat. 
		Al ejecutar la app manualmente desde el IDE Eclipse empleando como herramienta para desplegar el proyecto Tomcat, se modificó la fecha de actualización de subida de los ficheros pero no se actualización los datos de las vistas.
	\item Solucionar problema en la actualización de los ficheros csv.
		 Examinar y arreglar la causa por la cual los ficheros no se estaban actualizando. Para ello, se llevo a cabo varias modificaciones en las clases fachada de los datos.
	\item Cambiar configuración SonarCloud. 
		Se excluirá los ficheros que no se deseen examinar en el análisis de la calidad del código como los ficheros propios de Vaadin, los ficheros CSS, entre otros. Anteriormente el análisis que se había realizado sobre \textbf{SonarCloud} no estaba teniendo en cuenta los códigos en Java, esto es debido a que el análisis automático de SonarCloud no es compatible con Java. Se deberá especificar donde se encuentra el código fuente que se desea examinar y los archivos binarios de Java.
	\item Investigar cómo realizar el Login a través de UBUVirtual. 
		Uno de los requisitos es realizar un Login que permita autentificarse con el correo de la UBU o similares, por lo que, se realizará una búsqueda sobre cómo realizar esta conexión. Finalmente la opción que se escogió fue \href{https://firebase.google.com/}{Firebase}, un servicio de backend que dispone de SDKs fáciles de usar y bibliotecas de IU ya elaboradas.
	\item Investigar cómo importar los ficheros CSV y XLS a Heroku.
		Se puede subir los ficheros mediante Amazon S3 (o otro almacenamiento en la nube que se pueda conectar con Heroku) y, a través de Skyvia importar los datos (volcarlos) en la base de datos (Heroku Postgresql). O simplemente subir los ficheros en el .war y cuando se actualicen modificarlos, esta será la opción que se empleará.
	\item Prueba - Login con Microsoft.
		Se probará el código de ejemplo para iniciar sesión mediante Microsoft en aplicaciones web en Java. Siguiendo el tutorial de la página de \href{https://docs.microsoft.com/en-us/azure/active-directory/develop/quickstart-v2-java-webapp}{Microsoft}. Se registró la app en \textbf{Azure} y se siguió el tutorial de ejemplo \href{https://portal.azure.com/#blade/Microsoft_AAD_RegisteredApps/ApplicationsListBlade/quickStartType/JavaQuickstartPage/sourceType/docs}{Azure}. No se consiguió realizar el login ya que se necesitan permisos que no dispongo.
	\item Añadir más extensiones a gitignore. 
		Para evitar que se suban ficheros no deseados se incluyeron más extensiones a ignorar en el fichero gitignore. 
	
\end{itemize}

Se puede ver el transcurso de las tareas del Sprint en las siguientes gráficas\ref{fig:Sprint6_Graficos}.

\imagenflotante{Sprint6_Graficos}{Gráfica Control chart- Sprint 6}{0.9}

\imagenflotante{Sprint6_GraficoBurndown}{Gráfica Burndown- Sprint 6}{0.9}

\subsection{Sprint 7 (23/03/2021 - 13/04/2021)}
Análisis de la calidad del código de los proyectos presentados en 2020 con SonarCloud. Investigar cómo realizar el login con Firebase. Búsqueda y comienzo de proceso de migración de versión de Vaadin.

Las tareas realizadas fueron:
\begin{itemize}
	\tightlist
	\item Forks de todos los proyectos presentados en 2020.
		Se realizará un fork de todos los proyectos que se desean analizar y se añadirá al principio del nombre de los trabajos "UBU-TFG" para identificarlos. 
	\item Instalar SonarCloud CLI.
		Se realizará la instalación de SonarCloud en local siguiendo la documentación de \href{https://sonarcloud.io/documentation/analysis/scan/sonarscanner/}{SonarCloud} 
	\item SonarCloud - Analizar proyecto Gestión Aulas Informática
		Al ser un proyecto \textbf{Maven} en Java se realizará el análisis empleando el fichero pom.xml como se especifica en \href{https://github.com/dbo1001/Gestor-TFG-2021/issues/70}{Github}.
	\item SonarCloud - Analizar proyecto UBUMonitor Clustering
		Este proyecto no se pudo analizar, porque aparecía un error con respecto a la versión de java que empleaba. El error se puede apreciar en la \href{https://github.com/dbo1001/Gestor-TFG-2021/issues/72}{issue de Github} vinculada a esta tarea. 
	\item SonarCloud - Analizar proyecto Medidor estadístico metajuego Magic The Gathering
		Al ser un proyecto en Java se tuvo que realizar el análisis manual cómo se explica en las \href{https://github.com/dbo1001/Gestor-TFG-2021/issues/71}{tarea del proyecto de Github}.
	\item SonarCloud - Analizar proyecto TourPlanner-FrontEnd-Cliente
		En \href{https://github.com/dbo1001/Gestor-TFG-2021/issues/74}{Github} se puede ver el proceso que se llevo a cabo para realizar el análisis.
	\item SonarCloud - Analizar proyecto UBUVoiceAssistant.
		Se puede ver en detalle cómo se realizó el análisis del proyecto en la tarea de \href{https://github.com/dbo1001/Gestor-TFG-2021/issues/76}{Github} asociada.
	\item SonarCloud - Analizar proyecto LogScope.
		En la issue de \href{https://github.com/dbo1001/Gestor-TFG-2021/issues/75}{Github} se puede ver los pasos que se realizaron para analizar el trabajo.
	\item SonarCloud - Analizar proyecto PruebaNetExtractor.
		Al ser un proyecto en Python se empleo el análisis automático. En la \href{https://github.com/dbo1001/Gestor-TFG-2021/issues/79}{issue de Github} se explica cómo realizar el análisis automático.
	\item SonarCloud - Analizar proyecto Reserva aulas informática
		Se realizó el análisis automático como se indica en la tarea de \href{https://github.com/dbo1001/Gestor-TFG-2021/issues/80}{Github}.
	\item SonarCloud - Analizar proyecto MetrominutoWeb.
		Al igual que en los anteriores casos, se realizó el análisis automático de la calidad del código. En \href{https://github.com/dbo1001/Gestor-TFG-2021/issues/81}{Github} se puede ver el proceso que se llevo a cabo.
	\item SonarCloud - Analizar proyecto Sentinel.
		Como el proyecto emplea lenguajes compatibles con el análisis automático de SonarCloud, no se requerirá de modificaciones en el repositorio. Se puede obtener más información en \href{https://github.com/dbo1001/Gestor-TFG-2021/issues/82}{Github}.
	\item SonarCloud - Analizar proyecto CENIEH and Ariadne.
		En \href{https://github.com/dbo1001/Gestor-TFG-2021/issues/84}{Github} se puede ver el proceso que se llevo a cabo para realizar el análisis.
	\item SonarCloud - Analizar proyecto Plataforma de text mining sobre repositorios.
		Se realizon el anáisis automático como se especifica en la \href{https://github.com/dbo1001/Gestor-TFG-2021/issues/83}{tarea de Github correspondiente}.
	\item SonarCloud - Analizar proyecto UBUEstelas.	
		En este proyecto se empleo Gradle para el análisis del código, en la siguiente \href{https://github.com/dbo1001/Gestor-TFG-2021/issues/77}{tarea de Github} se detalla cómo se realizó.
	\item Investigar cómo migrar de versión de Vaadin. 
		En \href{https://github.com/dbo1001/Gestor-TFG-2021/issues/70}{Github} se puede ver el proceso que se llevo a cabo para realizar el análisis.
	\item SonarCloud - Analizar proyecto XRayDetector.
		Al ser un proyecto en Python bastó con realizar el análisis automático de SonarCloud. Se puede observar cómo se realizó en \href{https://github.com/dbo1001/Gestor-TFG-2021/issues/90}{Github}.
	\item SonarCloud - Analizar proyecto Jellyfish Forecast.
		Al igual que el caso anterior, es un proyecto en Python. En \href{https://github.com/dbo1001/Gestor-TFG-2021/issues/85}{la tarea de Github} asociada se puede ver cómo se crea el análisis.
	\item SonarCloud - Analizar proyecto Análisis Comercial Urbano
		En la \href{https://github.com/dbo1001/Gestor-TFG-2021/issues/86}{issue de Github} relacionada con este apartado se muestra el resultado del análisis de la calidad del digo con SonarCloud.
	\item SonarCloud - Analizar proyecto Iris classifier.
		En \href{https://github.com/dbo1001/Gestor-TFG-2021/issues/92}{Github} se explica los pasos seguidos para hacer el análisis.
	\item SonarCloud - Analizar proyecto Asistente de programación C.
		Esta tarea se encuentra explicada en la \href{https://github.com/dbo1001/Gestor-TFG-2021/issues/88}{issue de Github} se puede ver el proceso que se llevo a cabo para realizar el análisis.
	\item SonarCloud - Analizar proyecto Flutter Serpiente.
		Este proyecto fue eliminado del análisis de SonarCloud porque solo se consiguió que se analizará el lenguaje Kotlin y no el lenguaje principal, \textbf{Dart}. En la \href{https://github.com/dbo1001/Gestor-TFG-2021/issues/89}{tarea de Github} se proporcionan enlaces de fuentes donde se da información acerca de este lenguaje y cómo analizarlo. 
	\item Instalación NodeJS.
		Se modificó el pom.xml según el \href{https://vaadin.com/docs/v10/mpr/introduction/step-1-migration-guide}{manual de migración con mpr a Vaadin 14}. Según la documentación de Vaadin 14 se requiere tener instalado en el ordenador \href{https://nodejs.org/en/download/}{NodeJs}. En la \href{https://github.com/dbo1001/Gestor-TFG-2021/issues/97}{issue de Github} se ilustra un problema que surgió referente a las dependencias. 
	\item SonarCloud - Analizar proyecto Blockchain en una cadena de distribución de productos.
		Al igual que alguno de los proyecto anteriores, se realizo un análisis automático del proyecto. El resultado se puede ver en la \href{https://github.com/dbo1001/Gestor-TFG-2021/issues/87}{tarea de Github} asociada a esta tarea.
	\item Crear proyecto en Firebase.
		Para crear el proyecto de Firebase se creó una cuenta en el mismo y se procedió a añadir el proyecto desde la \href{https://console.firebase.google.com/?pli=1}{consola de Firebase} siguiendo los pasos ue se ilustran en la \href{https://github.com/dbo1001/Gestor-TFG-2021/issues/98}{issue de Github} asociada a esta tarea.
	\item Migración de versión de Vaadin
		Se decidió migrar la versión de Vaadin a la 14, debido a que la versión que se estaba empleando, Vaadin 7, ya tenía soporte y no era compatible con Java 11. Para realizar a migración se intentó incorporar \textbf{MPR} lo cual permite ejecutar la aplicación original en Vaadin 7 dentro de \textbf{Vaadin 14}. Se siguió la \href{https://vaadin.com/docs/v14/tools/mpr/introduction/step-1-maven-v7}{documentación de MPR en Vaadin} y se tomó como ejemplo el siguiente \href{https://github.com/OlliTietavainenVaadin/mpr-demo/tree/v7}{repositorio de demostración}.
	\item SonarCloud - Analizar proyecto Estudio de herramientas para realidad aumentada.
		No se consiguió analizar el lenguaje principal del proyecto, por lo que fue descartado.
	\item SonarCloud - Analizar proyecto ARBUBU.
		Al intentar analizar el proyecto daba un error que se puede ver en la ilustración compartida en la \href{https://github.com/dbo1001/Gestor-TFG-2021/issues/73}{issue} de Github. Este proyecto se dejará pendiente de revisión.
	\item SonarCloud - Analizar proyecto Free Connect.
		No se consigue analizar el lenguaje principal del proyecto por lo que queda pendiente de eliminación o descarte.
	\item Memoria - Objetivos del proyecto 
		Se desarrolla el apartado de Objetivos del proyecto de la memoria.
	\item Anexo - Especificación de Requisitos
		Se comienza a documentar en la parte de la Especificación de Requisitos de los anexos.
\end{itemize}

Se puede apreciar el progreso de las tareas del Sprint 7 en las siguientes gráficas\ref{fig:Sprint7_Graficos}.

\imagenflotante{Sprint7_Graficos}{Gráfica Control chart- Sprint 7}{0.9}

\imagenflotante{Sprint7_GraficoBurndown}{Gráfica Burndown- Sprint 7}{0.9}

\subsection{Sprint 8 (13/04/2021 - 04/05/2021)}
Continuación del proceso de migración el proyecto a Vaadin 14. Creación de la primera release (versión 0.6) con la app incluyendo la posibilidad de subir tanto .csv, como .xls, y su correspondiente actualización de las vistas.

Las tareas realizadas fueron:
\begin{itemize}
	\tightlist
	\item Revisión de memoria y anexos.
		Realización de correcciones recomendadas por el tutor de la documentación en LaTeX.
	\item Actualización subida de ficheros.
		Se llevaron a cabo varias modificaciones en el código para que se realizará correctamente la actualización de las vistas con los nuevos datos, tanto xls como csv.
	\item Incorporación Firebase para el Login. 
		Se siguió con la integración del login con \textbf{Firebase}.
	\item Migración a Vaadin 14.
		Continuación del proceso de migración del proyecto a Vaadin 14 para lo cual se debió de investigar y sustituir múltiples elementos que ya no existían en la nueva versión.
	\item Creación Release 0.6. 
		Primera versión con la aplicación en la versión con Vaadin 7 con la posibilidad de subir tanto ficheros csv como xls. 
\end{itemize}

Se puede ver el transcurso de las tareas del Sprint en las gráficas\ref{fig:Sprint8_Graficos}.

\imagenflotante{Sprint8_Graficos}{Gráfica Control chart- Sprint 8}{0.9}

\imagenflotante{Sprint8_GraficoBurndown}{Gráfica Burndown- Sprint 8}{0.9}

\subsection{Sprint 9 (04/05/2021 - 11/05/2021)}
Finalización del proceso de migración del proyecto a Vaadin 14 y despliegue de la aplicación en su nueva versión en \textbf{Heroku}.

Las tareas realizadas fueron:
\begin{itemize}
	\tightlist
	\item Investigar nueva API para realizar las gráficas del Histórico.
		Al cambiar de versión de Vaadin, \textbf{JFreeChart}, el componente empleado para realizar las gráficas del histórico, deja de ser compatible. Se realiza una búsqueda, en \href{https://vaadin.com/directory}{Vaadin Directory}, de componentes que puedan sustituirlo. Las dos mejores opciones encontradas son \href{https://vaadin.com/components/vaadin-charts}{Vaadin Chart}, la cual es de pago por lo que es descartada y, \href{https://apexcharts.com/}{ApexChart}, la opción elegida.
	\item Continuar la migración a Vaadin 14.
		Se concluye la migración de la app Gestor-TFG-2021 a Vaadin 14. Se pueden encontrar los enlaces de las páginas de documentación sobre el proceso y proyectos de referencia en la \href{https://github.com/dbo1001/Gestor-TFG-2021/issues/104}{issue de Github}.
	\item Actualización componentes de las Vistas.
		Se implementaron diversas modificaciones en las vistas para conseguir que su aspecto fuese similar a su versión anterior.
	\item Despliegue del proyecto en Heroku.
		Se vuelve a desplegar el proyecto en \href{https://gestor-tfg-2021.herokuapp.com/}{Heroku} con su versión en Vaadin 14 en su versión con Java 8.
	\item Modificación Login.
		Se prosigue la integración del login con Firebase.
	
\end{itemize}

En la siguiente ilustración se muestra la realización de las tareas del Sprint con respecto al tiempo\ref{fig:Sprint9_Graficos}.

\imagenflotante{Sprint9_Graficos}{Gráfica Control chart- Sprint 9}{0.9}

\imagenflotante{Sprint9_GraficoBurndown}{Gráfica Burndown- Sprint 9}{0.9}

\subsection{Sprint 10 (11/05/2021 - 18/05/2021)}
Realizar despliegue en Java 11 con la nueva versión en Vaadin 14. Continuar con el nuevo login con Firebase. Introducir mejoras en el código.

Las tareas realizadas fueron:
\begin{itemize}
	\tightlist
	\item Pruebas de actualización de ficheros en el proyecto en el despliegue de heroku. 
		Se probará a actualizar varios ficheros csv y xls para comprobar que se realiza correctamente. También se realizo una prueba con el fichero con extensión xls obtenido por los tutores en la anterior reunión para probar. Los test realizados se documentaron en la \href{https://github.com/dbo1001/Gestor-TFG-2021/issues/114}{tarea de Github} asociada a esta tarea.
	\item Despliegue del proyecto en Vaadin 14 en Heroku. 
		Se subió el proyecto en Java 11 a \href{https://gestor-tfg-2021.herokuapp.com/}{Heroku} según se indica en la \href{https://github.com/dbo1001/Gestor-TFG-2021/issues/112}{issue de Github}.Donde también se explica los cambios implementados para usar Java 11.
	\item Actualizar obtención ranking por cursos.
		La obtención de los cursos para el ranking se obtenía de la vista del proyecto ($N2_Proyecto$). Se cambiará para que obtenga el curso a partir de la fecha de presentación y asignación del Histórico ($N3_Historico$).
	\item Realizar mejoras en el código.
		Se realizará diversos cambios para aumentar la información y calidad del código, cómo la introducción de más información en el logger, actualización de los filtros empleados en las tablas (Grid), eliminación warnings y imports no usados.
	\item Cambio de versión a Java 11 en el workflow Maven CI/CD.
		En los últimos commits no se pasaron exitosamente los test de la Integración continua, debido a que se cambio la aplicación a Java 11, en el pom.xml, pero no en el workflow. Esto se solucionará cambiando la versión de Java (java-version) en el workflow, en el fichero github-ci.yml concretamente.
	\item Creación release 0.8
		Creación nueva release con la aplicación en \textbf{Java 11} y Vaadin 14.
	\item Actualizar styles del proyecto.
		Modificaciones estéticas de la aplicación para conseguir un resultado más similar a la versión anterior de la aplicación, con Vaadin 7. En estos cambios destacan, por ejemplo, el cambio del estilo del texto, tablas o títulos.
	
\end{itemize}

Se puede ver el tiempo qué se tardó en realizar las tareas y valores estadísticos del Sprint 10\ref{fig:Sprint10_Graficos}.

\imagenflotante{Sprint10_Graficos}{Gráfica Control chart- Sprint 10}{0.9}

\imagenflotante{Sprint10_GraficoBurndown}{Gráfica Burndown- Sprint 10}{0.9}

\subsection{Sprint 11 (18/05/2021 - 01/06/2021)}
Realizar login con base de datos Firestore. Añadir comprobaciones antes de acceder a la vista de la actualización de ficheros para que no se pueda acceder si no hay un usuario logueado. Modificaciones estéticas de la app. Cambiar la vista al iniciar la app a la vista de InformationView. Añadir en la vista de Métricas los análisis de calidad de código de SonarCloud.

Las tareas realizadas fueron:
\begin{itemize}
	\tightlist
	\item Continuar con Login con Firebase.
		En la \href{https://github.com/dbo1001/Gestor-TFG-2021/issues/122}{tarea de Github} vinculada a este apartado se detalla los pasos realizados.
	\item Actualizar README.
		Actualizar enlace al despliegue de Heroku y introducir más información acerca de la app y cómo ejecutarlo.
	\item Añadir reglas de seguridad en Firestore.
		Tras crear la base de datos Firestore se añadirán nuevas reglas de seguridad para impedir su modificación a usuarios no permitidos. Para realizar las modificaciones se siguió la documentación de \href{https://firebase.google.com/docs/firestore/security/insecure-rules}{Firebase} cómo se ve en la \href{https://github.com/dbo1001/Gestor-TFG-2021/issues/130}{issue}.
	\item Añadir iconos faltantes en la app.
		Se buscó iconos equivalentes a los que anteriormente había en las vistas de la aplicación, ya que al actualizar de versión de Vaadin ya no existen.
	\item Actualizar Footer.
		Añadir iconos en el Footer, los nombres faltantes (y sus respectivos correos de la ubu) y revisar fecha actualización ficheros. Se agregó, además de la fecha de actualización de los ficheros CSV, la fecha de actualización del archivo XLS.
	\item Renombrar ficheros XLS al subirse.
		Se modificó la lógica de la vista de actualización de ficheros (UploadView) para que se pudiese subir el fichero XLS con cualquier nombre y fuese la propia app la que lo renombrará con el nombre requerido. En el caso de los ficheros csv, se indicará en la vista de UploadView cómo deben llamarse: $N1_Documento$, $N1_Norma$, $N1_Tribunal$,$N2_Alumno$,$N2_Proyecto$ y $N3_Historico$.
	\item Añadir seguridad con Spring boot.
		Para evitar que se pueda acceder a la vista de la actualización de ficheros sin haber iniciado sesión previamente, se comenzó introduciendo Spring boot security pero, finalmente se encontró una forma más sencilla y se descarto el uso de esta opción.
	\item Modificar UploadView para verificar si hay un usuario logueado.
		Se añadió un método que comprueba y controla, antes de entrar al evento de UploadView, si hay algún usuario logueado. En caso contrario, redirige al login para que el usuario inicie sesión. 
	
\end{itemize}

En la siguiente gráfica se ilustrará el transcurso de las tareas del Sprint 11\ref{fig:Sprint11_Graficos}.

\imagenflotante{Sprint11_Graficos}{Gráfica Control chart- Sprint 11}{0.9}

\imagenflotante{Sprint11_GraficoBurndown}{Gráfica Burndown- Sprint 11}{0.9}

\subsection{Sprint 12 (01/06/2021 - 08/06/2021)}
Reemplazar el login existente por un login empleando \textbf{UbuVirtual} con Moodle. Añadir enlace en el botón de métricas de la navegación a SonarCloud, a los análisis de los proyectos. Realizar mejoras de código y documentar. Revisar análisis de proyectos realizados anteriormente con SonarCloud y crear nuevos con los proyectos entregado en 2019.

Las tareas realizadas fueron:
\begin{itemize}
	\tightlist
	\item Crear licencia.
		Crear fichero con licencia (LICENSE.md).
	\item Anexos - Especificación Requisitos.
		Incorporación de mejorar y ampliación del apartado Especificación Requisitos.
	\item Anexos - $D_Manual_programado$ - Estructura de directorios.
		Añadir el apartado Estructura de directorios, donde se especifica el contenido de cada carpeta del repositorio del proyecto.
	\item Cambiar el color de la celda en función del Ranking.
		Se intentó colorear las celdas del \textbf{rankings de percentiles} en función de su valor para dar así una mejor visibilidad. Se intentó realizar cómo se explica en diversos foros de la página de Vaadin pero, no se consiguió que se aplicarán los cambios sobre las celdas especificados en los estilos en css.
	\item Añadir plugin de cobertura en pom.xml.
		Añadir plugin que se empleaba en la aplicación anterior, con Vaadin 7, para incorporar documentos de medición de la cobertura de la app.
	\item Cambiar url del enlace a SonarCloud.
		Especificar la url en el apartado de métricas de la barra de acceso a las vistas a los \href{https://sonarcloud.io/organizations/dbo1001/projects}{proyectos en Sonarcloud}.
	\item Añadir test para lograr una mayor cobertura.
		Modificación de los test y incorporación de nuevos.
	\item Comentar y mejorar calidad código.
		Se revisará los comentarios y se mejorará la calidad del código, teniendo en cuenta el análisis realizado en la siguiente \href{https://github.com/dbo1001/Gestor-TFG-2021/issues/142}{issue de Analizar la app de Gestor-TFG-2021 con SonarCloud} . 
	\item Validar el fichero subido.
		Verificar el estado de la app tras subir un fichero incorrecto.
	\item Fork todos los proyectos entregados en 2019.
		Se realizará un Fork de todos los proyectos con FechaPresentacion 2019.
	\item SonarCloud - Analizar proyecto AVC.
		Se intentará analizar el proyecto correspondiente al TFG Asistente Virtual para la Comunicación, se puede ver los pasos realizados en \href{https://github.com/dbo1001/Gestor-TFG-2021/issues/147}{Github}.
	\item SonarCloud - Analizar proyecto UBUMonitor Clustering .
		Al realizar el análisis del proyecto surgieron varios errores debido a la falta de referencias de ciertas librerías necesarias. Al ser un proyecto Maven, en el pom.xml deberían estar referenciadas. Para intentar arreglar este problema, se modifico el pom.xml pero, no se consiguió solucionar el error. Se pueden ver los errores y los apsos seguidos en \href{https://github.com/dbo1001/Gestor-TFG-2021/issues/154}{Github}.
	\item SonarCloud - Analizar proyecto HealthApp.
		Se trata de un trabajo en Python, por tanto, se usó el análisis automático cómo puede verse en la \href{https://github.com/dbo1001/Gestor-TFG-2021/issues/153}{issue de Github}.
	\item SonarCloud - Analizar proyecto SmartBeds.
		Al igual que el proyecto anterior, se usó el análisis automático ya que usa Python. En la \href{https://github.com/dbo1001/Gestor-TFG-2021/issues/148}{issue de Github} se explica cómo se realizó.
	\item SonarCloud - Analizar proyecto Ububooknet.
		Se hizo uso del análisis automático cómo se detalla en la \href{https://github.com/dbo1001/Gestor-TFG-2021/issues/159}{tarea} asociada de Github.
	\item SonarCloud - Analizar proyecto InterpretacionEscalas UBU.
		El proyecto Interpretación de Escalas UBU se encuentra programado en su mayoría por Java, por lo que se necesito especificar los archivos fuente y binarios. En la \href{https://github.com/dbo1001/Gestor-TFG-2021/issues/163}{issue} de Github viene explicado este proceso.
	\item SonarCloud - Analizar proyecto Monitor-en-tiempo-real-de-un-sistema-de-fabricacion-aditiva para Octoprint.
		En la \href{https://github.com/dbo1001/Gestor-TFG-2021/issues/164}{tarea} de Github correspondiente a este apartado viene detallado el proceso de análisis.
	\item SonarCloud - Analizar proyecto PCVN.
		Se realizó el análisis de la calidad del código de este proyecto como se describe en la \href{https://github.com/dbo1001/Gestor-TFG-2021/issues/162}{issue} de Github.
	\item SonarCloud - Analizar proyecto NoisyNER
		Se trata de un trabajo en su mayoría en Python por lo que se realizará el análisis automático. Se puede ver cómo se realizó en la \href{https://github.com/dbo1001/Gestor-TFG-2021/issues/160}{issue} vinculada con esta tarea.
	\item Renombrar proyectos con UBU-TFG
		Se les añadirá "UBU TFG" al principio del nombre del repositorio para facilitar su búsqueda. Para modificar los nombres se acceder al apartado "Settings" > "Options" del repositorio y, una vez ahí, modificar el nombre.
	\item SonarCloud - Analizar proyecto Detección de Pallets mediante láser
		El proceso de está tarea viene detallado en \href{https://github.com/dbo1001/Gestor-TFG-2021/issues/158}{Github}.
	\item SonarCloud - Analizar proyecto Plataforma para el Análisis de Trayectorias Semánticas
		El análisis del TFG Plataforma para el Análisis de Trayectorias Semánticas se realizó como se describe en la \href{https://github.com/dbo1001/Gestor-TFG-2021/issues/157}{issue de Github}.
	\item SonarCloud - Analizar proyecto Sistema Recomendación TFG
		El proceso viene explicado en la \href{https://github.com/dbo1001/Gestor-TFG-2021/issues/156}{issue} de Github.
	\item SonarCloud - Analizar proyecto Datos públicos
		Se analizó el proyecto Aplicación Web para la recopilación, tratamiento y visualización de datos públicos 2 con SonarCloud como se indica en la \href{https://github.com/dbo1001/Gestor-TFG-2021/issues/155}{issue}.
	\item SonarCloud - Analizar proyecto Sistema Información sobre Matriculación
		Para hacer una evaluación de la calidad del código se realizaron los pasos indicados en la \href{https://github.com/dbo1001/Gestor-TFG-2021/issues/152}{issue} de Github.
	\item SonarCloud - Analizar proyecto Amazon-Scraper
		Se analizará el proyecto Extracción y procesamiento de datos de Amazon para su utilización en un estudio de marketing SonarCloud. \href{https://github.com/dbo1001/Gestor-TFG-2021/issues/151}{}
	\item SonarCloud - Analizar proyecto SmartBeds 1
		Al igual que en proyectos anteriores, los procedimientos seguidos se encuentran explicados en \href{https://github.com/dbo1001/Gestor-TFG-2021/issues/149}{Github}.
	\item SonarCloud - Analizar proyecto Aproximación hacia la identificación automática de lesión de ictus en TC craneal
		Esta tarea se encuentra detallada en la \href{https://github.com/dbo1001/Gestor-TFG-2021/issues/146}{tarea} de Github vinculada a esta.
	\item Revisar proyectos SonarCloud
		Se repasó los proyectos que se habían analizado previamente en SonarCloud para verificar que no se habían analizando lenguajes no deseados u otros errores. En la \href{https://github.com/dbo1001/Gestor-TFG-2021/issues/123}{tarea} de Github se enumeran as mejoras realizadas.
	\item Analizar la app de Gestor-TFG-2021 con SonarCloud
		Se volvió a realizar la medición de la calidad del código de la aplicación web con SonarCloud. Al emplear \textbf{Java 11} ya no hubo problemas de compatibilidad como pasaba anteriormente. En \href{https://github.com/dbo1001/Gestor-TFG-2021/issues/142}{Github} se explica cómo se analizó el proyecto \textbf{sistinf}.
	\item Actualizar despliegue de la app en Heroku
		Se actualizó el despliegue en \href{https://gestor-tfg-2021.herokuapp.com/}{Heroku} con los nuevos cambios introducidos. Una de las modificaciones más significativas fue la autentificación del login con el moodle de UbuVirtual.
	\item Añadir Login verificado con Moodle
		Se realizó modificaciones en el Login para realizar la verificación del usuario mediante el Moodle de \textbf{UbuVirtual}. Se tomará como referencia el proyecto \href{https://github.com/yjx0003/UBUMonitor}{UBUMonitor} y la documentación de Moodle. También se contó con el asesoramiento del profesor Raúl Marticorena Sánchez.
	\item Cambiar el banner para que aparezca el nombre de la app Gestor-TFG-2021
		Cuando se ejecutaba la app salia la cabecera de Apexchart, por lo que, se cambió para que figurará el nombre del proyecto "Gestor-TFG-2021". El banner se generó desde la página \href{https://devops.datenkollektiv.de/banner.txt/index.html}{Sprint Boot banner.txt generator}.
	\item Crear enlace en el botón de navegación Métricas a SonarCloud
		Se reemplazó la ventana de Métricas por un enlace que redirige a los análisis de los proyectos en SonarCload.
	\item Añadir una coma entre los nombres de los tutores en las tablas
		Se incluyó una coma entre los nombres de los tutores y alumnos en las tablas de Proyectos Activos y Histórico.
	\item Eliminar icono de la vista de Información
		Se suprimió el icono de participantes de la vista de InformationView para mejorar la estética de esa vista.
	
\end{itemize}

Se puede ver el desarrollo del Sprint 12 en la ilustración mostrada previamente\ref{fig:Sprint12_Graficos}.

\imagenflotante{Sprint12_Graficos}{Gráfica Control chart- Sprint 12}{0.9}

\imagenflotante{Sprint12_GraficoBurndown}{Gráfica Burndown- Sprint 12}{0.9}

\subsection{Sprint 13 (08/06/2021 - 18/06/2021)}
Revisión de la autenticación con \href{https://moodle.org/}{Moodle} y añadir test para comprobar la conexión por \textbf{webServices}.

Las tareas realizadas fueron:
\begin{itemize}
	\tightlist
	\item Añadir test para verificar el login a un moodle de ejemplo
		Crear test que verifiquen el funcionamiento de las funciones de inicio de sesión, obtención de cursos y comprobación de permisos de actualización con uno de las páginas \textbf{Moodle} de ejemplo.
	\item Modificar Login.
		Modificar Login para solamente mirar los permisos de la asignatura de \textbf{Trabajos de fin de grado}, en lugar de todos los cursos.
	
\end{itemize}

A continuación, se expondrán dos ilustración para mostrar el trascurso del Sprint y el desarrollo de las tareas\ref{fig:Sprint13_Graficos}.

\imagenflotante{Sprint13_Graficos}{Gráfica Control chart- Sprint 13}{0.9}

\imagenflotante{Sprint13_GraficoBurndown}{Gráfica Burndown- Sprint 13}{0.9}

\section{Estudio de viabilidad}
\subsection{Viabilidad económica}
En este apartado se detallan los costes que llevaría realizar este proyecto.

\subsubsection{Coste del personal}

\subsubsection{Coste hardware}
Referente a los costes del equipo utilizado en el desarrollo del trabajo. Teniendo en cuenta el precio del ordenador empleado de aproximadamente 700 euros.

\subsubsection{Coste software}
Referente a los costes de las herramientas software no gratuitas empleadas en el proyecto. Como es el caso del Sistema Operativo Windows o el Microsoft Office 365.

\subsection{Viabilidad legal}
En este apartado se detallaran las licencias de cada dependencia que se ha utilizado en el proyecto

\tablaSmallSinColores{Dependencias del proyecto}{ l | l | l }{dependencias}
{\textbf{Software} & \textbf{Licencia} \\}{
	Vaadin & Apache License 2.0 \\
	Vaadin Maven Plugin & Apache License 2.0 \\
}