\capitulo{3}{Conceptos teóricos}

Se explicaran algunos términos importantes para la comprensión del proyecto.

\section{\textit{Framework}}
Un \textbf{\textit{framework}}~\cite{framework_definicion}, también denominado marco de trabajo, es una estructura conceptual y tecnología estandarizada que facilita el desarrollo de aplicaciones web mediante un soporte basado en programas,bibliotecas, lenguajes, herramientas, prácticas, criterios, etc.

\section{\textit{Frontend}}
El \textbf{\textit{frontend}}~\cite{frontend_backend} es la parte de una aplicación web o programa que interactúa de forma directa con el usuario. Algunos de los lenguajes más usados para el desarrollo de frontend son: JavaScript, HTML y CSS.

\section{\textit{Backend}}
El \textbf{\textit{backend}}~\cite{frontend_backend} es la parte que se relaciona con la capa de datos, es decir, la base de datos y el servidor. Los usuarios no pueden acceder a ella debido a que contiene lógica de la aplicación. Para el desarrollo del backend se emplean lenguajes como Java y Python.

\section{Desarrollo \textit{full stack}}
El desarrollo \textbf{\textit{full stack}} es la unión del desarrollo del lado del cliente, \textit{frontend}, y el lado del servidor, \textit{backend}.

\section{Calidad del Software}
Hace referencia a una serie de especificaciones que debe cumplir un producto para cumplir con los requisitos establecidos, es decir, lograr la satisfacción del cliente. A pesar de ser un concepto muy amplio y subjetivo se puede calcular empleando estándares como por ejemplo el estándar \href{https://www.iso.org/home.html}{ISO} (\emph{International Organization for Standardization}) el cual describe la calidad como una serie de propiedades y características que debe cumplir un producto o servicio para cumplir con las necesidades del usuario o cliente. Se tendrá en cuenta la etapa del desarrollo del proyecto para determinar la calidad de un proyecto. 

Para garantizar la calidad del producto se emplean estándares de calidad.Tanto a\textbf{ nivel de proceso}, basados en la realización de acciones frecuentemente durante el desarrollo del software, como en el \textbf{producto final}, como podría ser por ejemplo los estándares de documentación y de código. 

Para realizar la evaluación de la calidad del software se evalúan, generalmente, empleando los puntos:
\begin{itemize}
	\item \textbf{Funcionalidad}: se define como la capacidad de un producto software para satisfacer las necesidades descritas en circunstancias normales.
	\item \textbf{Fiabilidad}: habilidad del software para mantener el nivel de prestaciones al usarse en condiciones especificas.
	\item \textbf{Seguridad}: se define como la capacidad de un producto software para satisfacer las necesidades descritas en circunstancias normales.
	\item \textbf{Usabilidad}: cualidad del software para dar facilidad de uso y aprendizaje al usuario.
	\item \textbf{Mantenibilidad}: habilidad de un producto software, bajo determinadas condiciones de uso, de conservarse su estado y sufrir modificaciones.
	\item \textbf{Eficiencia}: capacidad  para cumplir con las especificaciones con la menor cantidad de recursos posible.
	\item \textbf{Portabilidad}: propiedad para adaptarse a cualquier sistema o entorno.
\end{itemize}

En el caso de la evaluación del proceso de creación del software, una posible garantía de la calidad es la \textbf{gestión ágil}, aunque sólo es recomendable en proyectos pequeños. Consiste en la realización de pruebas sobre el código, durante la realización del desarrollo, para la detección de defectos o errores con la mayor brevedad posible. Es posible automatizar este proceso a través de ciertas herramientas como la \textbf{Integración continua}.

