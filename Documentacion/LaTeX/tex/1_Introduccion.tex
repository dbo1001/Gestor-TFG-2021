\capitulo{1}{Introducción}
El proyecto se basará en mejorar la actual aplicación del Gestor de TFG, utilizada en el grado de Ingeniería Informática. La aplicación actual se trata de una actualización previa del TFG denominado \href{https://github.com/jfb0019/Gestor-TFG-2016}{GII15.9 Gestión de trabajos fin de grado v2.0}. 

La aplicación requería de un login autenticado para la verificación de los usuarios en el acceso a la vista de la actualización de datos. Por lo que en la mejora se integró un \textbf{login} para la comprobación de las credenciales y los permisos de acceso mediante \textbf{UbuVirtual}.

Otra carencia que padecía era la falta de \textbf{compatibilidad con ficheros con múltiples hojas}, lo cual permitiría subir todos los datos de las vistas a la vez. Este problema fue resulto con la integración de una API de Excel, denominada \textbf{Fillo}, que permitía la obtención de los datos de un fichero de varias hojas de datos a través de lenguaje SQL.

El proyecto estaba codificado con \textbf{versiones obsoletas}, como en el caso Vaadin, una plataforma para el desarrollo de aplicaciones Java. Estos problemas fueron resueltos en esta mejora, reemplazando la gran mayoría de elementos de la aplicación por otros más nuevos y \textbf{actualizando las versiones de Vaadin y Java}. Con la actualización de los componentes se logró, a su vez, un mejor diseño de la aplicación.

\section{Estructura de la memoria}
La memoria consta de los siguientes apartados:

\begin{itemize}
	\item \textbf{Introducción:} presentación de la solución propuesta en el proyecto. 
	\item \textbf{Objetivos del proyecto:}  exposición de los  objetivos generales, técnicos y personales del proyecto.
	\item \textbf{Conceptos teóricos:} explicación de los términos teóricos necesarios para la comprensión y el desarrollo del proyecto.
	\item \textbf{Técnicas y herramientas:} definición de las técnicas metodológicas y las herramientas de desarrollo empleadas para desarrollar el proyecto.
	\item \textbf{Aspectos relevantes del desarrollo:} breve explicación de los términos más importantes durante el desarrollo del proyecto.
	\item \textbf{Trabajos relacionados:} descripción de los trabajos y proyectos asociados con la gestión de trabajos de fin de grado o master (TFG/TFM).
	\item \textbf{Conclusiones y líneas de trabajo futuras:} resolución obtenida al concluir el proyecto y descripción de posibles futuras líneas de trabajo o mejoras.
\end{itemize}