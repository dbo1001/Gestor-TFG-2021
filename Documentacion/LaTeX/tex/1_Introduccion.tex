\capitulo{1}{Introducción}
El proyecto se basará en mejora la actual aplicación del Gestor de TFG, utilizada en el grado de Ingeniería Informática. La aplicación actual se trata de una actualización previa en el TFG denominado \href{https://github.com/jfb0019/Gestor-TFG-2016}{GII15.9 Gestión de trabajos fin de grado v2.0}. 

Con el fin de solventar la necesidad de introducir la posibilidad de introducir la información de la aplicación con un formato que englobase varias hojas de datos, en lugar de tener que subir múltiples ficheros, y de validar a los usuarios a través de \href{https://ubuvirtual.ubu.es/}{UbuVirtual} para poder permitir su acceso a la actualización de los datos, se requirió de una mejora de la aplicación web.

\section{Estructura de la memoria}
La memoria consta de los siguientes apartados:

\begin{itemize}
	\item \textbf{Introducción:} Presentación de la solución propuesta en el proyecto. 
	\item \textbf{Objetivos del proyecto:}  Exposición de los  objetivos generales, técnicos y personales del proyecto.
	\item \textbf{Conceptos teóricos:} Explicación de los términos teóricos necesarios para la comprensión y el desarrollo del proyecto.
	\item \textbf{Técnicas y herramientas:} Definición de las técnicas metodológicas y las herramientas de desarrollo empleadas para desarrollar el proyecto.
	\item \textbf{Aspectos relevantes del desarrollo:} Breve explicación de los términos más importantes durante el desarrollo del proyecto.
	\item \textbf{Trabajos relacionados:} Descripción de los trabajos y proyectos asociados con la gestión de trabajos de fin de grado o master (TFG/TFM).
	\item \textbf{Conclusiones y líneas de trabajo futuras:} Resolución obtenida al concluir el proyecto y descripción de posibles futuras líneas de trabajo o mejoras.
\end{itemize}