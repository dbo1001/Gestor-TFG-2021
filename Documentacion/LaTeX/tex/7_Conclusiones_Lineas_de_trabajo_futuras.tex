\capitulo{7}{Conclusiones y Líneas de trabajo futuras}

\section{Conclusiones}
Una vez finalizado el trabajo se puede ver una evolución en cuanto a las herramientas empleadas y el manejo de las metodologías ágiles. Al realizar un seguimiento del proyecto a través de herramientas como GitHub y ZenHub que ha permitido un mejor control de las tareas.

Por otro lado, al ser un proyecto que propone una mejorar de una \textbf{aplicación ya existente} implica la \textbf{comprensión y aprendizaje del funcionamiento} del mismo. Al comienzo del desarrollo del proyecto se empleó mucho tiempo en conocer el código y aprender las nuevas herramientas que se iban a emplear, entre las cuales destaca Vaadin. Gracias a la cual, se ha logrado desarrollar los componentes de la aplicación sin grandes conocimientos en lenguajes de frontend como CSS o JavaScript. 
Otro punto a tener en cuenta es el empleó de Maven y Spring Boot que facilitan la configuración de las dependencias.

Al tener que \textbf{migrar de versión de Vaadin}, se tuvo que volver a aprender el funcionamiento de la navegación de Spring Boot y de los componentes de Vaadin 14, lo que requirió de mucho tiempo de desarrollo.

Otro punto donde se requirió gastar bastante tiempo fue en la \textbf{búsqueda de una API para la obtención de los datos de los ficheros con múltiples hojas}. A causa de la falta de opciones válidas que fuesen gratuitas. Se decidió cambiar de formato a xls para poder encontrar propuestas que pudiese servir.

El análisis de la \textbf{calidad del código} de los proyectos con lenguajes de programación soportados fue sencilla de realizar pero, los proyectos con lenguajes no compatibles requirieron de la comprensión de la estructura de los ficheros a través de la documentación del TFG. 

\section{Líneas de trabajo futuras}
Algunas de las posibles líneas de trabajo futuras del programa son:
\begin{itemize}
	\item Añadir la \textbf{posibilidad de solicitar la asignación de un TFG} a través de la aplicación web. Al escoger un TFG le llegaría un mensaje a los tutores designados, los cuales se podrían en contacto con el alumno interesado. 
	\item Modificar la columna de la tabla del Histórico, correspondiente al \textbf{ranking} de percentiles, para que \textbf{varíe el color del texto} según el valor de la celda. Introduciendo colores similares a los que se usa en SonarCloud, verde para el mejor resultado y rojo para el peor.
	\item Agregar un \textbf{usuario administrador} al cual será notificado cuando un usuario actualice la información de la página web. Para llevar un control de las actualizaciones que se realizan.
	\item Incluir la \textbf{internacionalización} de la aplicación.
\end{itemize}