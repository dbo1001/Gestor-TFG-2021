\capitulo{4}{Técnicas y herramientas}

Se van a presentar las técnicas metodológicas y las herramientas de desarrollo que se han utilizado para llevar a cabo el proyecto. En su mayoría se ha optado por elegir herramientas usadas previamente, como es el caso de: GitHub, Eclipse y GitHub Desktop.

\section{Lenguajes de programación}

\subsection{Java}
Java~\cite{pagina_java} es un lenguaje de programación orientado a objetos usado en el desarrollo de aplicaciones. Puede ser empleado tanto por sistemas operativos Unix como Windows debido a que es independiente de la plataforma. 
Consta de dos partes: 
\begin{itemize}
	\tightlist
	\item JRE (\emph{Java Runtime Environment}): es el entorno de ejecución o máquina virtual en la que se ejecuta las aplicaciones en Java. En la versión anterior de la aplicación se empleaba Java 8 pero, con la migración de Vaadin 14 se permitió actualizar a Java 11. En Java 11 la estructura cambió a una plataforma modular y ya no se proporciona el jre. 
	\item JDK (\emph{Java Development Kit}): es el API que distribuye Java para desarrollar aplicaciones empleando el lenguaje de programación. Anteriormente se empleaba Java 8 pero se actualizó a la versión 11, concretamente 11.0.1.0 de jdk. 
\end{itemize}

En la versión anterior se empleaba Java 8 pero, en este proyecto se llevará a cabo una migración a Vaadin 14 para poder actualizar la versión de \textbf{Java a la 11}.

\subsection{CSS}
CSS (Hojas de Estilo en Cascada o, en inglés, \emph{Cascading Stylesheets})~\cite{pagina_css} es un lenguaje de estilos empleado para el diseño gráfico de componentes o documentos en HTML.

\subsection{SQL}
SQL (Lenguaje de consulta estructurada o, en inglés, \emph{Structures Query Languaje})~\cite{pagina_sql} es un lenguaje de programación empleado en el tratamiento de datos y la relación entre ellos. Es usado tanto en la programación como en el diseño y administración de los datos, empleándose en gran medida en la recuperación y almacenamiento de la información en las Bases de datos.

\subsection{XML}
XML (Extensible Markup Language)~\cite{pagina_xml} es un lenguaje marcado que describe el conjunto de reglas para la codificación de documentos. Es simple, general y de facilidad de uso y, por lo tanto, se utiliza para varios servicios web.

\section{Herramientas de desarrollo}
\subsection{Maven}
Maven~\cite{pagina_maven} es una herramienta software con una arquitectura basada en plugins, desarrollada por Apache Software Foundation (ASF), usada para gestionar y construir proyectos en Java. 

Se configura el proyecto, a través de un \emph{Project Object Model}(POM) en formato XML (Extensible Markup Language), mediante dependencias con módulos y componentes externos. Además, incluye tareas como la compilación del código, su empaquetado, descarga e instalación de plugins. Existen plugins para trabajar con otros lenguajes como C/C++ y con el Framework .Net. 

\subsection{Vaadin}
Vaadin~\cite{pagina_vaadin} es una plataforma de código abierto para el desarrollo de aplicaciones web con Java. Permite el uso de lenguajes como HTML, CSS y JavaScript, etc. Permite diseñar la interfaz de usuario, medio donde el usuario interactúa con Java o TypeScript, empleando componentes propios predefinidos de Vaadin.

Requiere emplear un JDK (\textit{Java Development Kit}) y un IDE (Entorno de Desarrollo Integrado) como puede ser \href{https://www.eclipse.org/downloads/}{Eclipse}, \href{https://www.jetbrains.com/es-es/idea/}{IntelliJ Idea} o \href{https://netbeans.apache.org/}{NetBeans}.

Anteriormente se usaba Vaadin 7 pero, como la versión ya no contaba con soporte, se decidió realizar una migración a Vaadin 14. Con la actualización a Vaadin 14, la última versión estable hasta el momento, se pudo emplear JDK 11. Se empleó el IDE Eclipse ya que se había utilizado previamente.

\subsection{Spring Boot}
Spring Boot~\cite{pagina_springBoot} es un framework de código abierto para Java empleado para el desarrollo de aplicaciones web. Permite ejecutar aplicaciones en formato empaquetado sin necesidad de un servidor web embebido como \href{https://tomcat.apache.org/}{Tomcat}.

Requiere tener instalado un JDK (\textit{Java Development Kit}) de Java, como mínimo la versión 8, y Maven o Gradle. Adicionalmente se puede incluir un servidor como \href{https://tomcat.apache.org/}{Apache Tomcat}.

\subsection{Eclipse}
Eclipse~\cite{pagina_eclipse} es un IDE (entorno de Desarrollo Integrado) de código abierto y multiplataforma basado en Java. Proporciona diversas herramientas mediante plugins como la conexión con GitHub, PlantUml, etc. Y, múltiples frameworks como Spring boot, Android, Vaadin, Hibernate, JPA, Java EE, entre otros.

Como ya se había utilizado antes no requirió de un periodo de aprendizaje.

\subsection{Tomcat}
\href{https://tomcat.apache.org/download-90.cgi}{Tomcat}~\cite{pagina_tomcat} es un contenedor de servlets para el despliegue de aplicaciones web. 

\subsection{Fillo}
\href{https://codoid.com/fillo/}{Fillo}~\cite{pagina_fillo} es una API (Interfaz de Programación de Aplicaciones) de Excel en Java que permite realizar consultas en lenguaje SQL sobre ficheros en formato xls y xlsx. Soporta consultas, actualizaciones e inserciones en los ficheros.  Tras realizar una exhaustiva búsqueda se optó por emplear Fillo porque era la mejor opción gratuita y de fácil manejo que se encontró.

\subsection{ApexChart}
\href{https://apexcharts.com/}{ApexChart}~\cite{pagina_ApexChart} es una librería proveedora de gráficas interactivas para páginas web en formato SVG (Gráfico vectorial escalar).
Se trata de un proyecto de código abierto con licencia del MIT y su uso es gratuito en aplicaciones comerciales.

Se emplea en las gráficas de la vista del histórico de la aplicación.

\subsection{Moodle}
\href{https://moodle.org/}{Moodle}~\cite{pagina_Moodle} es una plataforma de aprendizaje para proporcionar a profesores, administradores y alumnos un sistema seguro, único y robusto donde crear entornos de aprendizaje personalizados.

Algunas de sus ventajas son: fácil de usar, gratuito, de confianza, multilingues y frecuentemente actualizado.

\subsection{Firebase}
\href{https://firebase.google.com/}{Firebase} es un servicio de backend que dispone de SDKs fáciles de usar y bibliotecas de IU ya elaboradas. 
Incorpora una base de datos online denominada ``Firestore Database'' y un sistema de autenticación de usuarios denominado ``Authentication'' de los cuales se hablará en la documentación.

\subsection{SonarCloud}
Es una plataforma gratuita para proyectos de código abierto y compatible con muchos lenguajes de programación, encargada de realizar análisis de calidad del código. También ofrece un servicio de pago para realizar los análisis privados.

Al realizar el análisis de la calidad del código se registran los diferentes tipos de problemas (o \emph{issues}):
\begin{itemize}
	\tightlist
	\item Error o fallo (\textbf{\emph{bug}}): error o fallo del software real.   
	\item \textbf{Vulnerabilidad}: puntos débiles de seguridad que pueden emplearse como foco de un ataque.
	\item Código sucio (\textbf{\emph{Code Smells}}): indica un problema en relación a la mantenibilidad del código puede deberse a qué se estén aplicando malas prácticas en el código o no se estén siguiendo los estándares de los lenguajes. Puede suponer un problema a largo plazo.
	\item \textbf{Código duplicado}: código empleado en diversas partes del código que hace que sea menos eficiente.
	\item \textbf{Cobertura}: se trata del porcentaje del código validado o probado por tests. Cuanto mayor es esta cifra menos probabilidades habrá de que ocasionen errores.	
\end{itemize}

Otros dos conceptos muy importantes en el ámbito de SonarCloud son:
\begin{itemize}
	\tightlist
	\item \textbf{Quality Profile}: son colecciones de reglas que se revisan en el análisis en función del lenguaje de programación predeterminado del proyecto.
	\item \textbf{Quality Gate}: condiciones que debe cumplir el proyecto para poder pasar a producción.
\end{itemize}

Se seleccionó SonarCloud como herramienta gestora de los análisis tras ser recomendada por los tutores.

\subsection{Heroku}
Heroku~\cite{pagina_heroku} es una plataforma en la nube con soporte en distintos lenguajes de programación que permite crear y desplegar aplicaciones. Permite añadir nuevas funcionalidades llamados \emph{add ons}, como es el caso de los gestores de bases de datos, por ejemplo PostgreSQL. 

El código se ejecuta dentro un contenedor denominado \emph{dyno}, los cuales garantizan la escalabilidad de la aplicación al ejecutarse más o menos dynos en función del número de conexiones que se establezcan. Además, incluye la posibilidad de conectar el código a través de GitHub.

Para emplear Heroku hay que registrarse y crear una cuenta, la cual puede ser:
\begin{itemize}
	\tightlist
	\item Gratuita: incluye las prestaciones básicas como el despliegue de una aplicación. Es la versión que se está empleando. 
	\item De pago: incorpora todas los servicios de Heroku.
\end{itemize}

Tras una búsqueda y análisis de las plataformas con hosting de despliegue gratuito se escogió Heroku porque proporcionaba más servicios y ventajas que el resto.

\section{Herramientas de documentación}
Se explicaran brevemente los programas empleados para realizar la documentación, tanto para el diseño de diagramas como para la redacción de la propia documentación.

\subsection{LaTeX}
LaTeX~\cite{pagina_latex} es un software libre para la composición de textos con una gran calidad tipográfica. Es empleado en gran medida para la creación de artículos, libros técnicos y tesis. 

En el proyecto se utilizó, recomendado por los tutores, \textbf{\href{https://miktex.org/}{Miktex}} como distribuidor de TeX junto con el editor de código abierto y multiplataforma, \textbf{\href{https://www.texstudio.org/}{TeXstudio}}. 

\subsection{Miktex} 
MikTeX~\cite{pagina_miktex} es una distribución de TeX/LaTeX multiplataforma que cuenta con un gestor de paquetes con la capacidad de instalar los componentes faltantes de Internet si fuese necesario.

\subsection{TeXstudio}
TeXstudio~\cite{pagina_texstudio} es un editor (IDE) de código abierto multiplataforma, similar al editor Texmaker, con la capacidad de editar, dar soporte, resaltar sintaxis, realizar revisiones ortográficas del código en LaTeX y visualizar el documento en un visor en formato pdf.

Se ha empleado para elaborar la documentación del proyecto haciendo uso de la plantilla proporcionada proporcionada.

\subsection{StartUML}
Se trata de una aplicación de escritorio para el diseño de diagramas UML, como es el caso de los diagramas de casos de uso, de clase o de secuencia. Se trabajó con este programa ya que anteriormente se había usado.

\subsection{Lucid app}
Es una plataforma online de diseño de diagramas gratuito y fácil de usar.

\subsection{PlantUML}
PlantUML~\cite{pagina_PlantUML} es una herramienta de código abierto para la creación de diagramas UML, a partir de lenguaje de texto sin formato. Se empleó para los diagramas de clase empleados en la documentación.

\section{Gestión del proyecto y control de versiones}
Para el control de versiones se ha optado por utilizar programas y plataformas ya conocidas.

\subsection{Metodología Scrum}
Scrum~\cite{pagina_Scrum} se trata de una metodología iterativa e incremental, empleada en el desarrollo y gestión ágil de proyectos. Describe una serie de buenas prácticas para promover la colaboración en equipos, consiguiendo mejores resultados que cumplen con las necesidades del cliente.

Comienza con la creación de una lista denominada \emph{product backlog} que servirá de guía para el desarrollo del producto. Se determinan las tareas (en ingles, \emph{issue}) que deben realizarse en cada intervalo de tiempo, al que se le nombra como \emph{sprint}. Generalmente un \emph{sprint} dura entre una a cuatro semanas. Se podrá añadir características a las tareas como la prioridad, la descripción del proceso, el tiempo estimado que llevará realizarla, el tipo, el responsable o persona asignada para dicha labor, entre otros.

Para supervisar el desarrollo se realizarán reuniones cortas, alrededor de quince minutos, y periódicas, donde se presentará el \emph{feedback} del producto al cliente. El equipo analizará el estado del proyecto y los problemas a futuro o ocasionados previamente. Según el estado en el que se encuentre el proyecto, se decidirá nuevamente una lista de tareas (\emph{product backlog}) para el siguiente \emph{sprint}. Se puede apreciar de forma resumida en la ilustración \ref{fig:Scrum_resumen}.

\imagenflotante{Scrum_resumen}{Esquema del proceso basado en Scrum~\cite{imagen_Scrum}}{1}

Este proceso continuará hasta la finalización del proyecto obteniendo el producto definitivo.

En el \textbf{proyecto del Gestor de TFG 2021}, se integró esta metodología a través de \textbf{Github}, donde aproximadamente cada quince días se realizaba una la reunión de seguimiento del TFG con los tutores. En las reuniones se revisaba los cambios introducidos en el anterior \emph{sprint}, los problemas que se pudiesen haber producido y se planteaba las siguientes \emph{issues} que debían desarrollarse en el siguiente \emph{sprint}.

\subsection{Integración Continua}
La CI (Integración continua)~\cite{pagina_CI} es una práctica de desarrollo de software que consiste en la búsqueda de posibles fallos en cada actualización del código del repositorio que se realice. Esta comprobación se realiza a través de la compilación del código fuente y la ejecución de los test. Para conseguir un proceso eficaz se deberán realizar integraciones del código periódicamente para poder detectar errores a la mayor brevedad posible.

Este proceso se realiza durante la fase de desarrollo o integración del software y conlleva un componente de automatización que en este caso se realizará mediante la opción \textbf{GitHub Actions} de Github.

\subsection{GitHub}
GitHub~\cite{pagina_github} es una plataforma de repositorios online colaborativos que permite llevar a cabo la gestión de proyectos y el control de versiones. A través de esta herramienta se realizó el seguimiento del proyecto integrando la \textbf{metodología ágil Scrum}.
 
La principal razón por la que se optó por Github fue porque ya se tenían ciertos conocimiento acerca de su funcionamiento. 

\subsection{ZenHub}
ZenHub~\cite{pagina_zenhub} es una plataforma de gestión de proyectos totalmente integrada en GitHub. Organiza las issues en el tablero \emph{canvas} según su estado: recién creadas, pendientes, en proceso, ya terminadas, etc. 

También incluye la posibilidad de generar gráficas para la obtención de información acerca del flujo del trabajo a lo largo del tiempo. Por ejemplo, el diagrama Burndown, empleado en los anexos de la documentación, muestra la comparativa del tiempo estimado para la realización de las tareas y el tiempo real que ha tomado en realizarse.

Se empleará esta herramienta junto con Github para realizar el seguimiento del proyecto empleando la Metodología Ágil Scrum.

\subsection{GitHub Desktop}
Github Desktop~\cite{pagina_github_desktop} simplifica la conexión con el repositorio GitHub sin necesidad de usar la línea de comandos de Git. A través de este programa se pueden realizar \emph{commit} y subirlos al GitHub mediante el comando \emph{push}, bajar los cambios realizados en el repositorio con \emph{pull}, entre otros.  