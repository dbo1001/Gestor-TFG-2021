\capitulo{4}{Técnicas y herramientas}

Se van a presentar las técnicas metodológicas y las herramientas de desarrollo que se han utilizado para llevar a cabo el proyecto. En su mayoría se he optado por elegir herramientas usadas previamente, como es el caso de: GitHub, Eclipse y GitHub Desktop.

\section{Código Fuente}

\subsection{Java}
Java~\cite{pagina_java} es un lenguaje de programación orientado a objetos usado en el desarrollo de aplicaciones. Puede ser empleado tanto por sistemas operativos Unix como Windows debido a que es independiente de la plataforma.

Consta de dos partes: 
JRE (\emph{Java Runtime Environment}): es el entorno de ejecución o máquina virtual en la que se ejecuta las aplicaciones en Java. En la versión anterior de la aplicación se empleaba Java 8 pero, con la migración de Vaadin 14 se permitió actualizar a Java 11. En Java 11 la estructura cambió a una plataforma modular y ya no se proporciona el jre. 
JDK (\emph{Java Development Kit}): es el API que distribuye Java para desarrollar aplicaciones empleando el lenguaje de programación. Anteriormente se empleaba Java 8 pero se actualizó a la versión 11, concretamente 11.0.1.0 de jdk. 

\subsection{CSS}
CSS (Hojas de Estilo en Cascada o, en inglés, Cascading Stylesheets)~\cite{pagina_css} es un lenguaje de estilos empleado para el diseño gráfico de componentes o documentos en HTML.

\subsection{XML}
XML (Extensible Markup Language)~\cite{pagina_xml} es un lenguaje marcado que describe el conjunto de reglas para la codificación de documentos. Es simple, general y de facilidad de uso y, por lo tanto, se utiliza para varios servicios web.

\section{Herramientas}
\subsection{Maven}
Maven~\cite{pagina_maven} es una herramienta software con una arquitectura basada en plugins, desarrollada por Apache Software Foundation(ASF), usada para gestionar y construir proyectos en Java. Configura el proyecto, a través de un \emph{Project Object Model}(POM) en formato XML (Extensible Markup Language), mediante dependencias con módulos y componentes externos. Además, incluye tareas como la compilación del código, su empaquetado, descarga e instalación de plugins. Existen plugins para trabajar con otros lenguajes como C/C++ y con el Framework .Net. 

\subsection{Vaadin}
Vaadin~\cite{pagina_vaadin} es una plataforma de código abierto para el desarrollo de aplicaciones web con Java. Permite el uso de lenguajes como HTML, CSS y JavaScript,etc.
Para usar Vaadin se requiere de por lo menos un JDK 8 (\textit{Java Development Kit}), un entorno de Desarrollo Integrado como Eclipse, NetBeans o IntelliJ Idea.

\section{Eclipse}
Eclipse~\cite{pagina_eclipse} es un IDE (entorno de Desarrollo Integrado) de código abierto y multiplataforma basado en Java. Proporciona diversas herramientas mediante plugins como la conexión con GitHub, PlantUml, etc. Y, múltiples frameworks como Spring boot, Android, Vaadin, Hibernate, JPA, Java EE, entre otros.

\subsection{LaTeX} es un software libre para la composición de textos con una gran calidad tipográfica. Es empleado en gran medida para la creación de artículos, libros técnicos y tesis. 

En el proyecto se empleo \textbf{\href{https://miktex.org/}{Miktex}} como distribuidor de TeX junto con el editor de código abierto y multiplataforma, \textbf{\href{https://www.texstudio.org/}{TeXstudio}}. 

\section{Técnicas}

\subsection{Metodología Scrum}
Se trata de una metodología iterativa e incremental, empleada en el desarrollo y gestión ágil de proyectos. Describe una serie de buenas prácticas para promover la colaboración en equipos, consiguiendo mejores resultados que cumplen con las necesidades del cliente.

Comienza con la creación de una lista denominada \emph{proguct backlog} que servirá de guía para el desarrollo del producto. Se determinan las tareas que deben realizarse en cada intervalo de tiempo, al que se le nombra como \emph{sprint}. Generalmente un \emph{sprint} durá entre una a cuatro semanas. Se podrá añadir características a las tareas como la prioridad, la descripción del proceso, el tiempo estimado que llevará realizarla, el tipo de tarea que es, el responsable o persona asignada para dicha tarea, entre otros. Las tareas se ordenarán según la prioridad de las tareas.

Para supervisar el desarrollo se realizarán reuniones cortas, alrededor de quince minutos, y periódicas, donde se presentará el \emph{feedback} del producto al cliente. El equipo analizará el estado del proyecto y los problemas a futuro o ocasionados previamente.Según el estado en el que se encuentre el proyecto, se decidirá nuevamente una lista de tareas (\emph{product backlog}) para el siguiente \emph{sprint}. Se puede apreciar de forma resumida en la siguiente ilustración \ref{fig:Scrum_resumen}.

\imagenflotante{Scrum_resumen}{Esquema del proceso basado en Scrum }{1.1}

Este proceso continuará hasta la finalización del proyecto obteniendo el producto definitivo.

En el \textbf{proyecto del Gestor de TFG 2021}, se integró esta metodología a través de \textbf{Github}, donde aproximadamente cada quince días se realizaba una la reunión de seguimiento del TFG con los tutores. En las reuniones se revisaba los cambios introducidos en el anterior \emph{sprint}, los problemas que se pudiesen haber producido y se planteaba las siguientes tareas que debían desarrollarse en el siguiente \emph{sprint}.

\subsection{Integración Continua}
Es una práctica de desarrollo de software que consiste en la búsqueda de posibles fallos en cada actualización del código del repositorio que se realice. Esta comprobación se realiza a través de la compilación del código fuente y la ejecución de los test. Para conseguir un proceso eficaz se deberán realizar integraciones del código frecuentemente para poder detectar errores con la mayor brevedad posible.

Este proceso se realiza durante la fase de desarrollo o integración del software y conlleva un componente de automatización que en este caso se realizará mediante la opción \textbf{GitHub Actions} de Github. 

\section{Gestión del proyecto y control de versiones}
Para el control de versiones se ha optado por utilizar programas y plataformas ya conocidas.

\subsection{GitHub}
GitHub~\cite{pagina_github} es una plataforma de repositorios online colaborativos que permite llevar a cabo la gestión de proyectos y el control de versiones. A través de esta herramienta se realizó el seguimiento del proyecto integrando la metodología ágil Scrum.
 
La principal razón por la que se optó por Github fue porque ya se tenían ciertos conocimiento acerca de su funcionamiento. 

\subsection{ZenHub}
ZenHub~\cite{pagina_zenhub} es una plataforma de gestión de proyectos totalmente integrada en GitHub. Organiza las issues en el tablero \emph{canvas} según su estado: recién creadas, pendientes, en proceso, ya terminadas, etc. 

También incluye la posibilidad de generar gráficas para la obtención de información acerca del flujo del trabajo a lo largo del tiempo. Por ejemplo, el diagrama Burndown, empleado en los anexos de la documentación, muestra la comparativa del tiempo estimado para la realización de las tareas y el tiempo real que ah tomado en realizarlas.

Se empleará está herramienta junto con Github para realizar el seguimiento del proyecto empleando la metodología ágil Scrum.

\subsection{GitHub Desktop}
Github Desktop~\cite{pagina_github_desktop} simplifica la tarea de conectar el repositorio GitHub sin necesidad de usar la línea de comandos de Git. A través de este programa se pueden realizar \emph{commit} y subirlos al GitHub mediante el comando \emph{push}, bajar los cambios realizados en el repositorio con \emph{pull}, entre otros. 