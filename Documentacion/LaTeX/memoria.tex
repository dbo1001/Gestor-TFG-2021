\documentclass[a4paper,12pt,twoside]{memoir}

% Castellano
\usepackage[spanish,es-tabla]{babel}
\selectlanguage{spanish}
\usepackage[utf8]{inputenc}
\usepackage[T1]{fontenc}
\usepackage{lmodern} % Scalable font
\usepackage{microtype}
\usepackage{placeins}

\RequirePackage{booktabs}
\RequirePackage[table]{xcolor}
\RequirePackage{xtab}
\RequirePackage{multirow}

% Links
\PassOptionsToPackage{hyphens}{url}\usepackage[colorlinks]{hyperref}
\hypersetup{
	allcolors = {red}
}

% Ecuaciones
\usepackage{amsmath}

% Rutas de fichero / paquete
\newcommand{\ruta}[1]{{\sffamily #1}}

% Párrafos
\nonzeroparskip

% Huérfanas y viudas
\widowpenalty100000
\clubpenalty100000

% Imagenes
\usepackage{graphicx}
\newcommand{\imagen}[2]{
	\begin{figure}[!h]
		\centering
		\includegraphics[width=0.9\textwidth]{#1}
		\caption{#2}\label{fig:#1}
	\end{figure}
	\FloatBarrier
}

\newcommand{\imagenflotante}[3]{ %Modificado para poder introducir el tamaño de la imagen(width)
	\begin{figure}[!h]
		\centering
		\includegraphics[width=#3\textwidth]{#1}
		\caption{#2}\label{fig:#1}
	\end{figure}
	\FloatBarrier
}


% El comando \figura nos permite insertar figuras comodamente, y utilizando
% siempre el mismo formato. Los parametros son:
% 1 -> Porcentaje del ancho de página que ocupará la figura (de 0 a 1)
% 2 --> Fichero de la imagen
% 3 --> Texto a pie de imagen
% 4 --> Etiqueta (label) para referencias
% 5 --> Opciones que queramos pasarle al \includegraphics
% 6 --> Opciones de posicionamiento a pasarle a \begin{figure}
\newcommand{\figuraConPosicion}[6]{%
  \setlength{\anchoFloat}{#1\textwidth}%
  \addtolength{\anchoFloat}{-4\fboxsep}%
  \setlength{\anchoFigura}{\anchoFloat}%
  \begin{figure}[#6]
    \begin{center}%
      \Ovalbox{%
        \begin{minipage}{\anchoFloat}%
          \begin{center}%
            \includegraphics[width=\anchoFigura,#5]{#2}%
            \caption{#3}%
            \label{#4}%
          \end{center}%
        \end{minipage}
      }%
    \end{center}%
  \end{figure}%
}

%
% Comando para incluir imágenes en formato apaisado (sin marco).
\newcommand{\figuraApaisadaSinMarco}[5]{%
  \begin{figure}%
    \begin{center}%
    \includegraphics[angle=90,height=#1\textheight,#5]{#2}%
    \caption{#3}%
    \label{#4}%
    \end{center}%
  \end{figure}%
}
% Para las tablas
\newcommand{\otoprule}{\midrule [\heavyrulewidth]}
%
% Nuevo comando para tablas pequeñas (menos de una página).
\newcommand{\tablaSmall}[5]{%
 \begin{table}
  \begin{center}
   \rowcolors {2}{gray!35}{}
   \begin{tabular}{#2}
    \toprule
    #4
    \otoprule
    #5
    \bottomrule
   \end{tabular}
   \caption{#1}
   \label{tabla:#3}
  \end{center}
 \end{table}
}

%
% Nuevo comando para tablas pequeñas (menos de una página).
\newcommand{\tablaSmallSinColores}[5]{%
 \begin{table}[H]
  \begin{center}
   \begin{tabular}{#2}
    \toprule
    #4
    \otoprule
    #5
    \bottomrule
   \end{tabular}
   \caption{#1}
   \label{tabla:#3}
  \end{center}
 \end{table}
}

\newcommand{\tablaApaisadaSmall}[5]{%
\begin{landscape}
  \begin{table}
   \begin{center}
    \rowcolors {2}{gray!35}{}
    \begin{tabular}{#2}
     \toprule
     #4
     \otoprule
     #5
     \bottomrule
    \end{tabular}
    \caption{#1}
    \label{tabla:#3}
   \end{center}
  \end{table}
\end{landscape}
}

%
% Nuevo comando para tablas grandes con cabecera y filas alternas coloreadas en gris.
\newcommand{\tabla}[6]{%
  \begin{center}
    \tablefirsthead{
      \toprule
      #5
      \otoprule
    }
    \tablehead{
      \multicolumn{#3}{l}{\small\sl continúa desde la página anterior}\\
      \toprule
      #5
      \otoprule
    }
    \tabletail{
      \hline
      \multicolumn{#3}{r}{\small\sl continúa en la página siguiente}\\
    }
    \tablelasttail{
      \hline
    }
    \bottomcaption{#1}
    \rowcolors {2}{gray!35}{}
    \begin{xtabular}{#2}
      #6
      \bottomrule
    \end{xtabular}
    \label{tabla:#4}
  \end{center}
}

%
% Nuevo comando para tablas grandes con cabecera.
\newcommand{\tablaSinColores}[6]{%
  \begin{center}
    \tablefirsthead{
      \toprule
      #5
      \otoprule
    }
    \tablehead{
      \multicolumn{#3}{l}{\small\sl continúa desde la página anterior}\\
      \toprule
      #5
      \otoprule
    }
    \tabletail{
      \hline
      \multicolumn{#3}{r}{\small\sl continúa en la página siguiente}\\
    }
    \tablelasttail{
      \hline
    }
    \bottomcaption{#1}
    \begin{xtabular}{#2}
      #6
      \bottomrule
    \end{xtabular}
    \label{tabla:#4}
  \end{center}
}

%
% Nuevo comando para tablas grandes sin cabecera.
\newcommand{\tablaSinCabecera}[5]{%
  \begin{center}
    \tablefirsthead{
      \toprule
    }
    \tablehead{
      \multicolumn{#3}{l}{\small\sl continúa desde la página anterior}\\
      \hline
    }
    \tabletail{
      \hline
      \multicolumn{#3}{r}{\small\sl continúa en la página siguiente}\\
    }
    \tablelasttail{
      \hline
    }
    \bottomcaption{#1}
  \begin{xtabular}{#2}
    #5
   \bottomrule
  \end{xtabular}
  \label{tabla:#4}
  \end{center}
}



\definecolor{cgoLight}{HTML}{EEEEEE}
\definecolor{cgoExtralight}{HTML}{FFFFFF}

%
% Nuevo comando para tablas grandes sin cabecera.
\newcommand{\tablaSinCabeceraConBandas}[5]{%
  \begin{center}
    \tablefirsthead{
      \toprule
    }
    \tablehead{
      \multicolumn{#3}{l}{\small\sl continúa desde la página anterior}\\
      \hline
    }
    \tabletail{
      \hline
      \multicolumn{#3}{r}{\small\sl continúa en la página siguiente}\\
    }
    \tablelasttail{
      \hline
    }
    \bottomcaption{#1}
    \rowcolors[]{1}{cgoExtralight}{cgoLight}

  \begin{xtabular}{#2}
    #5
   \bottomrule
  \end{xtabular}
  \label{tabla:#4}
  \end{center}
}


















\graphicspath{ {./img/} }

% Capítulos
\chapterstyle{bianchi}
\newcommand{\capitulo}[2]{
	\setcounter{chapter}{#1}
	\setcounter{section}{0}
	\chapter*{#2}
	\addcontentsline{toc}{chapter}{#2}
	\markboth{#2}{#2}
}

% Apéndices
\renewcommand{\appendixname}{Apéndice}
\renewcommand*\cftappendixname{\appendixname}

\newcommand{\apendice}[1]{
	%\renewcommand{\thechapter}{A}
	\chapter{#1}
}

\renewcommand*\cftappendixname{\appendixname\ }

% Formato de portada
\makeatletter
\usepackage{xcolor}
\newcommand{\tutor}[1]{\def\@tutor{#1}}
\newcommand{\course}[1]{\def\@course{#1}}
\definecolor{cpardoBox}{HTML}{E6E6FF}
\def\maketitle{
  \null
  \thispagestyle{empty}
  % Cabecera ----------------
\noindent\includegraphics[width=\textwidth]{cabecera}\vspace{1cm}%
  \vfill
  % Título proyecto y escudo informática ----------------
  \colorbox{cpardoBox}{%
    \begin{minipage}{.8\textwidth}
      \vspace{.5cm}\Large
      \begin{center}
      \textbf{TFG del Grado en Ingeniería Informática}\vspace{.6cm}\\
      \textbf{\LARGE\@title{}}
      \end{center}
      \vspace{.2cm}
    \end{minipage}

  }%
  \hfill\begin{minipage}{.20\textwidth}
    \includegraphics[width=\textwidth]{escudoInfor}
  \end{minipage}
  \vfill
  % Datos de alumno, curso y tutores ------------------
  \begin{center}%
  {%
    \noindent\LARGE
    Presentado por \@author{}\\ 
    en Universidad de Burgos --- \@date{}\\
    Tutor: \@tutor{}\\
  }%
  \end{center}%
  \null
  \cleardoublepage
  }
\makeatother

\newcommand{\nombre}{Diana Bringas Ochoa} 
\newcommand{\nombreTFG}{GII 20.09 Herramienta web repositorios de TFGII}

% Datos de portada
\title{\nombreTFG}
\author{\nombre}
\tutor{Álvar Arnaiz González y Carlos López Nozal}
\date{\today}

\begin{document}

\maketitle


\newpage\null\thispagestyle{empty}\newpage


%%%%%%%%%%%%%%%%%%%%%%%%%%%%%%%%%%%%%%%%%%%%%%%%%%%%%%%%%%%%%%%%%%%%%%%%%%%%%%%%%%%%%%%%
\thispagestyle{empty}


\noindent\includegraphics[width=\textwidth]{cabecera}\vspace{1cm}

\noindent D. Álvar Arnaiz González, profesor del departamento de Ingeniería Informática, área de Lenguajes y Sistemas Informáticos. Y, D. Carlos López Nozal, profesor del departamento de de Ingeniería Informática, área de Lenguajes y Sistemas Informáticos.

\noindent Expone:

\noindent Que la alumna D. \nombre, con DNI 79131451-C, ha realizado el Trabajo final de Grado en Ingeniería Informática titulado Gestor de TFG 2021. 

\noindent Y que dicho trabajo ha sido realizado por el alumno bajo la dirección del que suscribe, en virtud de lo cual se autoriza su presentación y defensa.

\begin{center} %\large
En Burgos, {\large \today}
\end{center}

\vfill\vfill\vfill

% Author and supervisor
\begin{minipage}{0.45\textwidth}
\begin{flushleft} %\large
Vº. Bº. del Tutor:\\[2cm]
D. Álvar Arnaiz González
\end{flushleft}
\end{minipage}
\hfill
\begin{minipage}{0.45\textwidth}
\begin{flushleft} %\large
Vº. Bº. del co-tutor:\\[2cm]
D. Carlos López Nozal
\end{flushleft}
\end{minipage}
\hfill

\vfill

% para casos con solo un tutor comentar lo anterior
% y descomentar lo siguiente
%Vº. Bº. del Tutor:\\[2cm]
%D. nombre tutor


\newpage\null\thispagestyle{empty}\newpage




\frontmatter

% Abstract en castellano
\renewcommand*\abstractname{Resumen}
\begin{abstract}
El proyecto propone evolucionar la aplicación web de Gestor de trabajos de fin de grado, una aplicación web para el manejo de procesos de oferta, búsqueda, asignación y evaluación de los TFG de la carrera de Ingeniería Informática en la Universidad de Burgos. 
Emplea como lenguaje de programación Java y emplea componentes gráficos del framework, Vaadin. 

Se requiere de mejoras en seguridad, la autenticación de usuarios, en el diseño gráfico con Vaadin y en la obtención de los datos, con la adición de la posibilidad de emplear un nuevo tipo de fichero de datos. 

También se solicita la realización de métricas mediante el análisis de la calidad del código en una plataforma online como SonarCLoud.
\end{abstract}

\renewcommand*\abstractname{Descriptores}
\begin{abstract}
Aplicación web, Java, Vaadin, trabajo de fin de grado (TFG), gestor de proyecto, Moodle, Fillo, Metodología ágil, Spring Boot, SonarCloud, Maven, ApexChart, Heroku \ldots
\end{abstract}

\clearpage

% Abstract en inglés
\renewcommand*\abstractname{Abstract}
\begin{abstract}
The project proposes to update the final degree project manager web application, which is in charge of managing the offer, search, assignment and evaluation processes of Computer Engineering career final projects career of the University of Burgos. 
It uses Java as the main programming languaje and graphical components of the Vaadin Framework. 

It requires improvements in security, like the user authentication, graphic design using Vaadin and data collection, such as the use of a new type of data file.

It are also requested to do metrics by analyzing the quality of the code on an online platform such as SonarCloud.
\end{abstract}

\renewcommand*\abstractname{Keywords}
\begin{abstract}
Web application, Java, Vaadin, final degree project, project manager, Moodle, Fillo, Agile methodology, Spring Boot, SonarCloud, Maven, ApexChart, Heroku \ldots
\end{abstract}

\clearpage

% Indices
\tableofcontents

\clearpage

\listoffigures

\clearpage

\listoftables
\clearpage

\mainmatter
\capitulo{1}{Introducción}
El proyecto se basará en mejorar la actual aplicación del Gestor de TFG, utilizada en el grado de Ingeniería Informática. La aplicación actual se trata de una actualización previa del TFG denominado \href{https://github.com/jfb0019/Gestor-TFG-2016}{GII15.9 Gestión de trabajos fin de grado v2.0}. 

La aplicación requería de un login autenticado para la verificación de los usuarios, en el acceso a la vista de la actualización de datos. Por lo que en la mejora se integró un \textbf{login} en el cual se comprobaban las credenciales y los permisos de acceso mediante \textbf{UbuVirtual}.

Otra carencia que padecía era la falta de \textbf{compatibilidad con ficheros con múltiples hojas}, lo cual permitiría subir todos los datos de las vistas a la vez. Este problema fue resulto con la integración de una API de Excel, denominada \textbf{Fillo}, que permitía la obtención de los datos de un ficheros de varias hojas de datos a través de lenguaje SQL.

El proyecto estaba codificado con \textbf{versiones obsoletas}, como en el caso Vaadin, una plataforma para el desarrollo de aplicaciones Java. Estos problemas fueron resueltos en esta mejora, reemplazando múltiples elementos de la aplicación por otros más nuevos y \textbf{actualizando las versiones de Vaadin y Java}. Con la actualización de los componentes se logró, a su vez, un mejor diseño de la aplicación.

\section{Estructura de la memoria}
La memoria consta de los siguientes apartados:

\begin{itemize}
	\item \textbf{Introducción:} Presentación de la solución propuesta en el proyecto. 
	\item \textbf{Objetivos del proyecto:}  Exposición de los  objetivos generales, técnicos y personales del proyecto.
	\item \textbf{Conceptos teóricos:} Explicación de los términos teóricos necesarios para la comprensión y el desarrollo del proyecto.
	\item \textbf{Técnicas y herramientas:} Definición de las técnicas metodológicas y las herramientas de desarrollo empleadas para desarrollar el proyecto.
	\item \textbf{Aspectos relevantes del desarrollo:} Breve explicación de los términos más importantes durante el desarrollo del proyecto.
	\item \textbf{Trabajos relacionados:} Descripción de los trabajos y proyectos asociados con la gestión de trabajos de fin de grado o master (TFG/TFM).
	\item \textbf{Conclusiones y líneas de trabajo futuras:} Resolución obtenida al concluir el proyecto y descripción de posibles futuras líneas de trabajo o mejoras.
\end{itemize}
\capitulo{2}{Objetivos del proyecto}
En este apartado se describen los objetivos del proyecto.

\section{Objetivos generales}

\begin{itemize}
	\item Permitir subir al usuario la información en un nuevo formato que permita subir múltiples hojas con datos.
	\item Mejorar el diseño gráfico de la aplicación.
	\item Posibilitar la validación de usuario a través del correo de la Universidad para acceder a la vista de la actualización de los datos de la aplicación web.
	\item Emplear la metodología ágil para realizar el seguimiento del desarrollo del proyecto.
\end{itemize}

\section{Objetivos técnicos}
\begin{itemize}
	\item Incorporar una API o plugin para incorporar otro tipo de datos (XLS).
	\item Integrar una herramienta que permita el login a través del correo de la Universidad de forma segura y con posibilidad de restringir o dar permisos de acceso a ciertos usuarios. 
	\item Utilizar GitHub para llevar a cabo el seguimiento del proyecto y control de versiones.
	\item Crear métricas de la calidad del código de los TFG.
	\item Crear la nueva capa de datos asociada al nuevo tipo de datos empleando patrones de diseño.
\end{itemize}
\capitulo{3}{Conceptos teóricos}

Se explicaran algunos términos importantes para la comprensión del proyecto.

\section{\textit{Framework}}
Un \textbf{\textit{framework}}~\cite{framework_definicion}, también denominado marco de trabajo, es una estructura conceptual y tecnológico estandarizada que facilita el desarrollo de aplicaciones web mediante un soporte basado en programas,bibliotecas, lenguajes, herramientas, prácticas, criterios, etc.

\section{\textit{Frontend}}
El \textbf{\textit{frontend}}~\cite{frontend_backend} es la parte de una aplicación web o programa que interactúa de forma directa con el usuario. Algunos de los lenguajes más usado para el desarrollo de frontend son: JavaScript, HTML y CSS.

\section{\textit{Backend}}
El \textbf{\textit{backend}}~\cite{frontend_backend} es la parte que se relaciona con la capa de datos, es decir, la base de datos y el servidor. Los usuarios no pueden acceder a ella debido a que contiene lógica de la aplicación. Para el desarrollo del backend se emplean lenguajes como Java y Python.

\section{Desarrollo \textit{full stack}}
El desarrollo \textbf{\textit{full stack}} es la unión del desarrollo del lado del cliente, \textit{frontend}, y el lado del servidor, \textit{backend}.
\capitulo{4}{Técnicas y herramientas}

Se van a presentar las técnicas metodológicas y las herramientas de desarrollo que se han utilizado para llevar a cabo el proyecto. En su mayoría se ha optado por elegir herramientas usadas previamente, como es el caso de: GitHub, Eclipse y GitHub Desktop.

\section{Lenguajes de programación}

\subsection{Java}
Java~\cite{pagina_java} es un lenguaje de programación orientado a objetos usado en el desarrollo de aplicaciones. Puede ser empleado tanto por sistemas operativos Unix como Windows debido a que es independiente de la plataforma. 
Consta de dos partes: 
\begin{itemize}
	\tightlist
	\item JRE (\emph{Java Runtime Environment}): es el entorno de ejecución o máquina virtual en la que se ejecuta las aplicaciones en Java. En la versión anterior de la aplicación se empleaba Java 8 pero, con la migración de Vaadin 14 se permitió actualizar a Java 11. En Java 11 la estructura cambió a una plataforma modular y ya no se proporciona el jre. 
	\item JDK (\emph{Java Development Kit}): es el API que distribuye Java para desarrollar aplicaciones empleando el lenguaje de programación. Anteriormente se empleaba Java 8 pero se actualizó a la versión 11, concretamente 11.0.1.0 de jdk. 
\end{itemize}

En la versión anterior se empleaba Java 8 pero, en este proyecto se llevará a cabo una migración a Vaadin 14 para poder actualizar la versión de \textbf{Java a la 11}.

\subsection{CSS}
CSS (Hojas de Estilo en Cascada o, en inglés, \emph{Cascading Stylesheets})~\cite{pagina_css} es un lenguaje de estilos empleado para el diseño gráfico de componentes o documentos en HTML.

\subsection{SQL}
SQL (Lenguaje de consulta estructurada o, en inglés, \emph{Structures Query Languaje})~\cite{pagina_sql} es un lenguaje de programación empleado en el tratamiento de datos y la relación entre ellos. Es usado tanto en la programación como en el diseño y administración de los datos, empleándose en gran medida en la recuperación y almacenamiento de la información en las Bases de datos.

\subsection{XML}
XML (Extensible Markup Language)~\cite{pagina_xml} es un lenguaje marcado que describe el conjunto de reglas para la codificación de documentos. Es simple, general y de facilidad de uso y, por lo tanto, se utiliza para varios servicios web.

\section{Herramientas de desarrollo}
\subsection{Maven}
Maven~\cite{pagina_maven} es una herramienta software con una arquitectura basada en plugins, desarrollada por Apache Software Foundation (ASF), usada para gestionar y construir proyectos en Java. 

Se configura el proyecto, a través de un \emph{Project Object Model}(POM) en formato XML (Extensible Markup Language), mediante dependencias con módulos y componentes externos. Además, incluye tareas como la compilación del código, su empaquetado, descarga e instalación de plugins. Existen plugins para trabajar con otros lenguajes como C/C++ y con el Framework .Net. 

\subsection{Vaadin}
Vaadin~\cite{pagina_vaadin} es una plataforma de código abierto para el desarrollo de aplicaciones web con Java. Permite el uso de lenguajes como HTML, CSS y JavaScript, etc. Permite diseñar la interfaz de usuario, medio donde el usuario interactúa con Java o TypeScript, empleando componentes propios predefinidos de Vaadin.

Requiere emplear un JDK (\textit{Java Development Kit}) y un IDE (Entorno de Desarrollo Integrado) como puede ser \href{https://www.eclipse.org/downloads/}{Eclipse}, \href{https://www.jetbrains.com/es-es/idea/}{IntelliJ Idea} o \href{https://netbeans.apache.org/}{NetBeans}.

Anteriormente se usaba Vaadin 7 pero, como la versión ya no contaba con soporte, se decidió realizar una migración a Vaadin 14. Con la actualización a Vaadin 14, la última versión estable hasta el momento, se pudo emplear JDK 11. Se empleó el IDE Eclipse ya que se había utilizado previamente.

\subsection{Spring Boot}
Spring Boot~\cite{pagina_springBoot} es un framework de código abierto para Java empleado para el desarrollo de aplicaciones web. Permite ejecutar aplicaciones en formato empaquetado sin necesidad de un servidor web embebido como \href{https://tomcat.apache.org/}{Tomcat}.

Requiere tener instalado un JDK (\textit{Java Development Kit}) de Java, como mínimo la versión 8, y Maven o Gradle. Adicionalmente se puede incluir un servidor como \href{https://tomcat.apache.org/}{Apache Tomcat}.

\subsection{Eclipse}
Eclipse~\cite{pagina_eclipse} es un IDE (entorno de Desarrollo Integrado) de código abierto y multiplataforma basado en Java. Proporciona diversas herramientas mediante plugins como la conexión con GitHub, PlantUml, etc. Y, múltiples frameworks como Spring boot, Android, Vaadin, Hibernate, JPA, Java EE, entre otros.

Como ya se había utilizado antes no requirió de un periodo de aprendizaje.

\subsection{Fillo}
\href{https://codoid.com/fillo/}{Fillo}~\cite{pagina_fillo} es una API (Interfaz de Programación de Aplicaciones) de Excel en Java que permite realizar consultas en lenguaje SQL sobre ficheros en formato xls y xlsx. Soporta consultas, actualizaciones e inserciones en los ficheros. 

Tras realizar una exhaustiva búsqueda se optó por emplear Fillo ya que era la mejor opción gratuita y de fácil manejo que se encontró.

\subsection{ApexChart}
\href{https://apexcharts.com/}{ApexChart}~\cite{pagina_ApexChart} es una librería proveedora de gráficas interactivas para páginas web en formato SVG (Gráfico vectorial escalar).
Se trata de un proyecto de código abierto con licencia del MIT y su uso es gratuito en aplicaciones comerciales.

Se emplea en las gráficas de la vista del histórico de la aplicación.

\subsection{Moodle}
\href{https://moodle.org/}{Moodle}~\cite{pagina_Moodle} es una plataforma de aprendizaje para proporcionar a profesores, administradores y alumnos un sistema seguro, único y robusto donde crear entornos de aprendizaje personalizados.

Algunas de sus ventajas son: fácil de usar, gratuito, de confianza, multilingues y frecuentemente actualizado.

\subsection{Firebase}
\href{https://firebase.google.com/}{Firebase} es un servicio de backend que dispone de SDKs fáciles de usar y bibliotecas de IU ya elaboradas. 
Incorpora una base de datos online denominada ``Firestore Database'' y un sistema de autenticación de usuarios denominado ``Authentication'' de los cuales se hablará en la documentación.

\subsection{SonarCloud}
Es una plataforma gratuita para proyectos de código abierto y compatible con muchos lenguajes de programación, encargada de realizar análisis de calidad del código. También ofrece un servicio de pago para realizar los análisis privados.

Al realizar el análisis de la calidad del código se registran los diferentes tipos de problemas (o \emph{issues}):
\begin{itemize}
	\tightlist
	\item Error o fallo (\textbf{\emph{bug}}): error o fallo del software real.   
	\item \textbf{Vulnerabilidad}: puntos débiles de seguridad que pueden emplearse como foco de un ataque.
	\item Código sucio (\textbf{\emph{Code Smells}}): indica un problema en relación a la mantenibilidad del código puede deberse a qué se estén aplicando malas prácticas en el código o no se estén siguiendo los estándares de los lenguajes. Puede suponer un problema a largo plazo.
	\item \textbf{Código duplicado}: código empleado en diversas partes del código que hace que sea menos eficiente.
	\item \textbf{Cobertura}: se trata del porcentaje del código validado o probado por tests. Cuanto mayor es esta cifra menos probabilidades habrá de que ocasionen errores.	
\end{itemize}

Otros dos conceptos muy importantes en el ámbito de SonarCloud son:
\begin{itemize}
	\tightlist
	\item \textbf{Quality Profile}: son colecciones de reglas que se revisan en el análisis en función del lenguaje de programación predeterminado del proyecto.
	\item \textbf{Quality Gate}: condiciones que debe cumplir el proyecto para poder pasar a producción.
\end{itemize}

Se seleccionó SonarCloud como herramienta gestora de los análisis tras ser recomendada por los tutores.

\subsection{Heroku}
Heroku~\cite{pagina_heroku} es una plataforma en la nube con soporte en distintos lenguajes de programación que permite crear y desplegar aplicaciones. Permite añadir nuevas funcionalidades llamados \emph{add ons}, como es el caso de los gestores de bases de datos, por ejemplo PostgreSQL. 

El código se ejecuta dentro un contenedor denominado \emph{dyno}, los cuales garantizan la escalabilidad de la aplicación al ejecutarse más o menos dynos en función del número de conexiones que se establezcan. Además, incluye la posibilidad de conectar el código a través de GitHub.

Para emplear Heroku hay que registrarse y crear una cuenta, la cual puede ser:
\begin{itemize}
	\tightlist
	\item Gratuita: incluye las prestaciones básicas como el despliegue de una aplicación. Es la versión que se está empleando. 
	\item De pago: incorpora todas los servicios de Heroku.
\end{itemize}

Tras una búsqueda y análisis de las plataformas con hosting de despliegue gratuito se escogió Heroku porque proporcionaba más servicios y ventajas que el resto.

\section{Herramientas de documentación}
Se explicaran brevemente los programas empleados para realizar la documentación, tanto para el diseño de diagramas como para la redacción de la propia documentación.

\subsection{LaTeX}
LaTeX~\cite{pagina_latex} es un software libre para la composición de textos con una gran calidad tipográfica. Es empleado en gran medida para la creación de artículos, libros técnicos y tesis. 

En el proyecto se utilizó, recomendado por los tutores, \textbf{\href{https://miktex.org/}{Miktex}} como distribuidor de TeX junto con el editor de código abierto y multiplataforma, \textbf{\href{https://www.texstudio.org/}{TeXstudio}}. 

\subsection{Miktex} 
MikTeX~\cite{pagina_miktex} es una distribución de TeX/LaTeX multiplataforma que cuenta con un gestor de paquetes con la capacidad de instalar los componentes faltantes de Internet si fuese necesario.

\subsection{TeXstudio}
TeXstudio~\cite{pagina_texstudio} es un editor (IDE) de código abierto multiplataforma, similar al editor Texmaker, con la capacidad de editar, dar soporte, resaltar sintaxis, realizar revisiones ortográficas del código en LaTeX y visualizar el documento en un visor en formato pdf.

Se ha empleado para elaborar la documentación del proyecto haciendo uso de la plantilla proporcionada proporcionada.

\subsection{StartUML}
Se trata de una aplicación de escritorio para el diseño de diagramas UML, como es el caso de los diagramas de casos de uso, de clase o de secuencia. Se trabajó con este programa ya que anteriormente se había usado.

\subsection{Lucid app}
Es una plataforma online de diseño de diagramas gratuito y fácil de usar.

\subsection{PlantUML}
PlantUML~\cite{pagina_PlantUML} es una herramienta de código abierto para la creación de diagramas UML, a partir de lenguaje de texto sin formato. Se empleó para los diagramas de clase empleados en la documentación.

\section{Gestión del proyecto y control de versiones}
Para el control de versiones se ha optado por utilizar programas y plataformas ya conocidas.

\subsection{Metodología Scrum}
Scrum~\cite{pagina_Scrum} se trata de una metodología iterativa e incremental, empleada en el desarrollo y gestión ágil de proyectos. Describe una serie de buenas prácticas para promover la colaboración en equipos, consiguiendo mejores resultados que cumplen con las necesidades del cliente.

Comienza con la creación de una lista denominada \emph{proguct backlog} que servirá de guía para el desarrollo del producto. Se determinan las tareas (en ingles, \emph{issue}) que deben realizarse en cada intervalo de tiempo, al que se le nombra como \emph{sprint}. Generalmente un \emph{sprint} dura entre una a cuatro semanas. Se podrá añadir características a las tareas como la prioridad, la descripción del proceso, el tiempo estimado que llevará realizarla, el tipo, el responsable o persona asignada para dicha labor, entre otros.

Para supervisar el desarrollo se realizarán reuniones cortas, alrededor de quince minutos, y periódicas, donde se presentará el \emph{feedback} del producto al cliente. El equipo analizará el estado del proyecto y los problemas a futuro o ocasionados previamente. Según el estado en el que se encuentre el proyecto, se decidirá nuevamente una lista de tareas (\emph{product backlog}) para el siguiente \emph{sprint}. Se puede apreciar de forma resumida en la ilustración \ref{fig:Scrum_resumen}.

\imagenflotante{Scrum_resumen}{Esquema del proceso basado en Scrum~\cite{imagen_Scrum}}{1}

Este proceso continuará hasta la finalización del proyecto obteniendo el producto definitivo.

En el \textbf{proyecto del Gestor de TFG 2021}, se integró esta metodología a través de \textbf{Github}, donde aproximadamente cada quince días se realizaba una la reunión de seguimiento del TFG con los tutores. En las reuniones se revisaba los cambios introducidos en el anterior \emph{sprint}, los problemas que se pudiesen haber producido y se planteaba las siguientes \emph{issues} que debían desarrollarse en el siguiente \emph{sprint}.

\subsection{Integración Continua}
La CI (Integración continua)~\cite{pagina_CI} es una práctica de desarrollo de software que consiste en la búsqueda de posibles fallos en cada actualización del código del repositorio que se realice. Esta comprobación se realiza a través de la compilación del código fuente y la ejecución de los test. Para conseguir un proceso eficaz se deberán realizar integraciones del código periódicamente para poder detectar errores a la mayor brevedad posible..

Este proceso se realiza durante la fase de desarrollo o integración del software y conlleva un componente de automatización que en este caso se realizará mediante la opción \textbf{GitHub Actions} de Github.

\subsection{GitHub}
GitHub~\cite{pagina_github} es una plataforma de repositorios online colaborativos que permite llevar a cabo la gestión de proyectos y el control de versiones. A través de esta herramienta se realizó el seguimiento del proyecto integrando la \textbf{metodología ágil Scrum}.
 
La principal razón por la que se optó por Github fue porque ya se tenían ciertos conocimiento acerca de su funcionamiento. 

\subsection{ZenHub}
ZenHub~\cite{pagina_zenhub} es una plataforma de gestión de proyectos totalmente integrada en GitHub. Organiza las issues en el tablero \emph{canvas} según su estado: recién creadas, pendientes, en proceso, ya terminadas, etc. 

También incluye la posibilidad de generar gráficas para la obtención de información acerca del flujo del trabajo a lo largo del tiempo. Por ejemplo, el diagrama Burndown, empleado en los anexos de la documentación, muestra la comparativa del tiempo estimado para la realización de las tareas y el tiempo real que ha tomado en realizarse.

Se empleará esta herramienta junto con Github para realizar el seguimiento del proyecto empleando la Metodología Ágil Scrum.

\subsection{GitHub Desktop}
Github Desktop~\cite{pagina_github_desktop} simplifica la conexión con el repositorio GitHub sin necesidad de usar la línea de comandos de Git. A través de este programa se pueden realizar \emph{commit} y subirlos al GitHub mediante el comando \emph{push}, bajar los cambios realizados en el repositorio con \emph{pull}, entre otros.  
\capitulo{5}{Aspectos relevantes del desarrollo del proyecto}

Este apartado pretende recoger los aspectos más interesantes del desarrollo del proyecto, comentados por los autores del mismo.
Debe incluir desde la exposición del ciclo de vida utilizado, hasta los detalles de mayor relevancia de las fases de análisis, diseño e implementación.
Se busca que no sea una mera operación de copiar y pegar diagramas y extractos del código fuente, sino que realmente se justifiquen los caminos de solución que se han tomado, especialmente aquellos que no sean triviales.
Puede ser el lugar más adecuado para documentar los aspectos más interesantes del diseño y de la implementación, con un mayor hincapié en aspectos tales como el tipo de arquitectura elegido, los índices de las tablas de la base de datos, normalización y desnormalización, distribución en ficheros3, reglas de negocio dentro de las bases de datos (EDVHV GH GDWRV DFWLYDV), aspectos de desarrollo relacionados con el WWW...
Este apartado, debe convertirse en el resumen de la experiencia práctica del proyecto, y por sí mismo justifica que la memoria se convierta en un documento útil, fuente de referencia para los autores, los tutores y futuros alumnos.

\section{Gestión del proyecto mediante metodología ágil}
Para realizar el seguimiento y control del proyecto se implemento la metodología ágil con herramientas como \href{https://github.com/}{GitHUb} y \href{https://www.zenhub.com/}{ZenHub}. 

Se realizaron reuniones se seguimiento cada, aproximadamente 15 días, donde se exponían los cambios realizados en el periodo anterior (denominado \emph{Sprint}) y las tareas que debían realizarse en el próximo periodo de tiempo. Esto permitía un mejor control del tiempo y de las tareas pendientes. 

Se introdujo el uso de la integración continua de forma automática a través de GitHub donde se compilaba y ejecutaban tests para verificar que no existían fallos en el código. Esto permitió 


\section{Integración de la API Fillo}
Uno de los objetivos era introducir la recuperación de datos con ficheros de múltiples hojas, concretamente en formato ods.
Para ello se investigó una alternativa similar al driver empleado para leer los ficheros csv. Primero se probó la opción de \textbf{\href{https://www.cdata.com/drivers/excel/jdbc/}{Microsoft Excel JDBC Driver}} con el cual se puede leer, escribir y actualizar Excel mediante JDBC. Sin embargo, está opción es de pago, por lo que fue descartada. 

Después se testeo dos drivers gratuitos, \href{https://odftoolkit.org/}{ODFDOM} y \href{http://www.jopendocument.org/}{JopenDocument}, los cuales eran bastante viejos y en desuso.

Como no se encontró ninguna solución válida, se decidió con los tutores cambiar el formato del fichero xls, ya que en este formato existían más opciones. Se volvió a realizar una búsqueda de drivers o APIs para realizar la conexión con ficheros xls y se encontraron varias opciones: 
\begin{itemize}
	\item \href{https://poi.apache.org/}{Apache POI}: permite, mediante el empleo de bibliotecas en Java puro, leer y escribir en archivos en formatos de Microsoft Office como Excel. Para verificar su funcionamiento, se incluyo en el proyecto de prueba ``HolaMundoVaadin'' y se realizaron diversas pruebas. Aunque funcionaba bien fue descartado ya que su inclusión requería de rehacer todas las funciones de obtención de datos y no se podría reutilizar código, empleado en la lectura de los ficheros csv.
	\item \href{https://code.google.com/archive/p/sqlsheet/}{SqlSheet}: driver en JDBC para a conexión con ficheros xls y xlsx basado en Apache POI. Aunque era una buena opción, fue eliminada porque solo permitía hacer consultas sencillas de todo el documento, en lugar de seleccionar los datos por columnas o según una condición. 
	\item \href{https://codoid.com/fillo/}{Fillo}: es una API(Interfaz de Programación de Aplicaciones) de Excel en Java que permite realizar consultas en lenguaje SQL sobre ficheros en formato xls y xlsx. Para testearlo se uso el proyecto de prueba ``HolaMundoVaadin'' y tras verificar su funcionamiento se incluyó en el proyecto principal. Se escogió esta opción porque exigía menos modificaciones y la posibilidad de utilizar de código.
\end{itemize}

Para realizar la integración de \textbf{\href{https://codoid.com/fillo/}{Fillo}} se separó el código, referente a la lectura de los datos en dos partes, una para cada tipo de fichero que la aplicación permite. Esto ocasionó la creación de nuevos tests para la comprobación de la obtención de la información en el nuevo tipo de fichero.

\section{Migración a Vaadin 14}
La aplicación empleaba Vaadin 7 pero, cómo ya no contaba con soporte y requería usar Java 8 se decidió actualizar Vaadin a la versión 14, la última versión estable.  

Para realizar la migración se intentó incorporar \textbf{MPR}(siglas en inglés,\emph{Multiplatform Runtime}) que ejecuta la aplicación original, en Vaadin 7, dentro de una aplicación en \textbf{Vaadin 14}. Para ello, se siguió la \href{https://vaadin.com/docs/v14/tools/mpr/introduction/step-1-maven-v7}{documentación de MPR en Vaadin} y se tomó como ejemplo el \href{https://github.com/OlliTietavainenVaadin/mpr-demo/tree/v7}{repositorio de demostración} mencionado en la documentación.

Tras intentar realizar la migración mediante \textbf{MPR}(siglas en inglés,\emph{Multiplatform Runtime}) y no conseguir resultados, se desistió y se comenzó a realizar la \textbf{migración de cero a Vaadin 14}. Se descargó uno de los proyectos de ejemplo de Vaadin 14, para tomarlo como referencia. 

La actualización de los componentes y la navegación conllevo realizar una búsqueda y aprendizaje de los componentes en Vaadin 14, ya que se apreciaba un gran cambio.

Otro apartado en el que se apreció un gran cambio fue en la navegación de la aplicación que se incluyó \textbf{Spring Boot} porque ofrecía una manera sencilla y rápida de ejecutar la aplicación sin necesidad de añadir un servidor web embebido.

La migración requirió más cambios de los esperados por lo que conllevo mucho tiempo realizarla.

\section{Autenticación de usuarios con UbuVirtual}

Uno de los requisitos era realizar un login que permita autentificarse con el correo de la Universidad de Burgos, para lo cual se realizó una recopilo posibles herramientas podían usarse para realizar la conexión y verificación del usuario.

Primero se intentó realizar \textbf{la conexión a través de Microsoft} como se indica en su \href{https://docs.microsoft.com/en-us/azure/active-directory/develop/quickstart-v2-java-webapp}{documentación sobre Azure} pero no se consiguió.

Después se probó con \href{https://firebase.google.com/}{Firebase}, con la opción de ``\textbf{Authentication}'' desde la cual se puede añadir usuarios y gestionar sus permisos. El problema era que requería de una persona que introdujera los datos y de realizar un proceso de codificación y descodificación de las contraseñas, por lo que no era viable. Con el caso de la base de datos online que tiene Firebase, ``\textbf{Firestore}'' ocurría lo mismo por lo que también se descarto.  

Se quería introducir un sistema que no necesitase de una persona para gestionarlo, por tanto, se optó por la autenticación mediante el \textbf{\href{https://moodle.org/}{moodle} de \href{https://ubuvirtual.ubu.es/}{UbuVirtual}}. 

Como consecuencia de esta modificación se tuvo que añadir mucho código y aprender cómo realizar la conexión y obtención de la información de moodle ya que nunca se había trabajo con esta plataforma.

\capitulo{6}{Trabajos relacionados}
Se nombraran algunos proyectos y aplicaciones similares o relacionados con la gestión de trabajos de fin de grado o master (TFG/TFM).

\section{Gestor-TFG-2016}
Es el proyecto que se debe mejorar. Es una aplicación Web para la gestión de TFG. Emplea como lenguaje de programación Java y Vaadin 7 como plataforma de código abierto para la interfaz web.

\section{GESTFG}
Es un sistema de gestión de TFG y TFM empleado en la Escuela Técnica Superior de Ingenierías Informática y Telecomunicaciones de Granada. Se empleo el framework Django. La plataforma administra la información relacionada con los TFGs, usuarios, las diferentes fases (asignación, evaluación), interacción entre los miembros asignados a un TFG (alumnos, tutores y tribunal), notificaciones vía correo electrónico, evaluación de los TFG, entre otras funcionalidades.

Se intento realizar el despliegue del proyecto para visualizar el proyecto pero ya no se encuentra operativo, seguramente debido a que no se ha realizado mantenimiento.







\capitulo{7}{Conclusiones y Líneas de trabajo futuras}

\section{Conclusiones}

\section{Líneas de trabajo futuras}
Algunas de las posibles líneas de trabajo futuras del programa son:
\begin{itemize}
	\item Incorporación de nuevos tipos de datos que se puedan emplear. 
\end{itemize}


\bibliographystyle{plain}
\bibliography{bibliografia}

\end{document}
